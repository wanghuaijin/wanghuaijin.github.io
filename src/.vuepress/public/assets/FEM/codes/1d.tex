
% This LaTeX was auto-generated from MATLAB code.
% To make changes, update the MATLAB code and republish this document.

\documentclass[11pt]{article}
\usepackage[utf8]{inputenc}
\usepackage[T1]{fontenc}
\usepackage{amsthm}
\usepackage{enumitem}
\usepackage{amssymb}
\usepackage{amsmath}
\usepackage{amsfonts}
\usepackage[version=4]{mhchem}
\usepackage{stmaryrd}
\usepackage{mathrsfs}
\usepackage{bm}
\usepackage{graphicx}
\usepackage[export]{adjustbox}
\graphicspath{ {./images/} }
\usepackage{algorithm}
\usepackage{algorithmic}
\usepackage{makecell}  % 表格换行






 \usepackage{hyperref}
\hypersetup{
    colorlinks=true,     % 启用颜色链接
    linkcolor=blue,     % 内部链接的颜色
    citecolor=red,      % 引用文献的颜色
    urlcolor=blue,       % URL链接的颜色
    linktoc=red,      % 不影响目录链接颜色
}
\usepackage[a4paper, top=1in, bottom=1in, left=1in, right=1in]{geometry}

\title{
{\bf \huge FEM: Basic Theory and Implementation}\\
{\bf \large 1-D  FEM for Elliptic Equation}
}
\author{Huaijin Wang}
\date{\today}


\begin{document}


\newtheorem{definition}{Definition}[section]
\newtheorem{property}{Property}[section]
\newtheorem{lemma}{Lemma}[section]
\newtheorem{theorem}{Theorem}[section]
\newtheorem{corollary}{Corollary}[section]
\newtheorem{remark}{Remark}[section]
\newtheorem{example}{Example}[]
\numberwithin{equation}{section}


\maketitle

\begingroup
\hypersetup{
    linkcolor=black,  % 将目录链接颜色设置为黑色
}
\tableofcontents
\endgroup


\newpage

\section{Introduction}

\noindent $\bullet$ Why Elliptic problem?

\noindent $\bullet$ Why 1D case?

\noindent $\bullet$ Why FEM?

\section{Elliptic Problem}
We consider the elliptic problem:



\subsection{Typical Model: Possion Equation with Homogeneous Dirichlet Boundary}
A two-point boundary value problem with homogeneous Dirichlet boundary condition:
\begin{equation}
\left\{
\begin{aligned}
-\frac{d^2 u}{d x^2} & =f(x), \quad x\in I:=(0,1), \\
u(0) & =u(1)=0,
\end{aligned}
\right.
\label{eq:1d-sp}
\end{equation}
where $f \in L^2(I)$. The problem \eqref{eq:1d-sp} is also called the \textit{strong problem}. Let $V:=H_0^1 (I) = \{v\in H^1(I): v(0)=v(1) = 0\}$. The restriction on boundary values makes sense due to the embedding theorem, which tells that 
\[
\forall v\in H^1(I), \ \exists \bar{v} \in C(\bar{I}) \ \text{s.t.}\ v= \bar{v} \ \text{a.e. in}\ I.
\]
The \textit{variational problem} (or known as \textit{weak problem}):
\begin{equation}
\left \{
\begin{aligned}
& \text{Find} \ u\in V \ \text{such that}\ \\
& \left( u^\prime, v^\prime \right)=(f, v), \quad \forall v \in V,
\end{aligned}
\right.
\label{eq:1d-wp}
\end{equation}
where $(\cdot, \cdot)$ stands for the $L^2(I)-$inner product. 
Let $J$ be the linear functional:
\[
J(v) = \frac{1}{2}  \left( u^\prime, v^\prime \right) - (f,v).
\]
Then the \textit{minimization problem}: 
\begin{equation}
\left \{
\begin{aligned}
& \text{Find}\ u\in V \ \text{such that}\ \\
& J(u) \leqslant J(v), \quad \forall v\in V.
\end{aligned}
\right.
\label{eq:1d-mp}
\end{equation}
The term {minimization problem} corresponds the "principle of minimum potential energy" in mechanics. It tells us that some of differential equations like \eqref{eq:1d-sp} may originates from minimizing the potential energy in some physical problems.

\begin{theorem}
Under proper regularity assumptions, the three problems above are equivalent:
~\\
1). the solution of \eqref{eq:1d-sp} is a solution of \eqref{eq:1d-wp};\\
2). the solution of \eqref{eq:1d-wp} is a solution of \eqref{eq:1d-mp}; \\
3). the solution of \eqref{eq:1d-mp} is a solution of \eqref{eq:1d-sp}.
\end{theorem}

The existence and uniqueness of these three problem can be considered respectively. 

For \eqref{eq:1d-sp}, its existence can be represented using Green's function (see \cite[p.35 Chapter 2, Theorem 12]{evans2010}), and the uniqueness is guaranteed by the \textit{strong maximum principle} of harmonic functions (see \cite[pp. 27-28, Theorem 4\&5]{evans2010}).

For \eqref{eq:1d-mp}, its existence and uniqueness are guaranteed by that $J$ is strongly convex and is a linear functional over a linear space.

For \eqref{eq:1d-wp}, its existence and uniqueness are guaranteed by the well known \textit{Lax-Milgram Lemma}, whose general description reads
\begin{lemma}[Lax-Milgram]
Let $V$ be a Hilbert space, endowed with the norm $\|\cdot \|_V$. Consider the problem: $\forall f\in V^\prime$,
\[
\left\{
\begin{aligned}
&\text{Find}\ u\in V, \ \text{such that} \\
& a(u,v) = <f,v>,\quad \forall v\in V,
\end{aligned}
\right.
\]
where $a(\cdot, \cdot): V\times V\to \mathbb{R}$ is a bilinear form. If furthermore, $a(\cdot,\cdot)$ satisfies
\[
\begin{aligned}
& \mathrm{Continuity:}\quad \exists \gamma > 0 \ \text{s.t.}\ |a(u,v)| \leqslant \gamma \|u\|_V \|v\|_V, \quad \forall u,v \in V, \\
& \mathrm{Coercivity:}\quad  \exists \alpha > 0 \ \text{s.t.}\ a(v,v) \geqslant \alpha \|v\|_V^2,\quad \forall v\in V .
\end{aligned}
\]
Then the problem admits a unique solution $u$, which satisfies
\[
\|u\|_V \leqslant \frac{1}{\alpha} \sup_{v\in V, v\neq 0} \frac{<f,v>}{\|v\|_V}.
\]
\end{lemma}

\begin{theorem}
Problem \eqref{eq:1d-wp} admits an unique solution.
\end{theorem}

\subsection{Other Boundary Conditions}



\section{$P1-$ FEM}
The finite element method (FEM) is a numerical technique, arguably the most robust and popular, for solving differential equations. FEM is a numerical method general based on the \textit{Galerkin approximation} (or \textit{Galerkin method} or \textit{Galerkin framework}), to approximate with constructing finite elements (piecewise approximation). Galerkin method is to approximate the weak problem with finite dimensional subspace constructed. For \eqref{eq:1d-wp}, 
\begin{equation}
\left \{
\begin{aligned}
&\text{Find}\ u_h \in V_h \ \text{such that} \\
& \left(\frac{d u_h}{d x}, \frac{d v_h}{d x}\right)=(f, v_h), \quad \forall v_h \in V_h,
\end{aligned}
\right.
\label{eq:1d-gm}
\end{equation}
where $V_h$ is a finite dimensional subspace of $V$.


We divide the interval $[0,1]$ into $N+2$ grid
\[
0=x_0 < x_1 <\cdots < x_N < x_{N+1} = 1.
\]
We denote the subintervals $I_j = [x_{j-1}, x_j]$ for $1\leqslant j \leqslant N+1$, with length $h_j = x_j-x_{j-1}$. Let $h = \max_{1\leqslant j \leqslant N+1} h_j$. The mesh size $h$ is used to measure how fine the partition is.

We define the finite element space
\[
V_h= \left \{v \in C[0,1]:   v \text { is linear on each subinterval } I_j, \text { and } v(0)=v(1)=0\right\} .
\]
\begin{theorem}
$V_h \subset V$.
\end{theorem}
\begin{proof}
It is sufficient to show that for any $v \in V_h$ we have $v\in H^1(I)$, i.e.,
\[
\int_0^1 \frac{dv}{dx} \phi \mathrm{d} x = -\int_0^1 v \frac{d\phi}{dx} \mathrm{d} x,\quad \forall \phi \in C_0^\infty (I).
\]
In fact,
\[
\begin{aligned}
\int_0^1 \frac{dv}{dx} \phi \mathrm{d} x & = \sum_{j=1}^{N+1} \int_{I_j} \frac{dv}{dx} \phi \mathrm{d} x = \sum_{j=1}^{N+1} \left(\phi(x_j) v(x_j) - \phi(x_{j-1}) v(x_{j-1}) - \int_{I_j} v \frac{d\phi}{dx}  \mathrm{d} x \right) \\
& = \phi(1) v(1) - \phi(0) v(0) - \sum_{j=1}^{N+1} \int_{I_j} v \frac{d\phi}{dx} \mathrm{d} x = - \int_0^1 v \frac{d\phi}{dx} \mathrm{d} x.
\end{aligned}
\]
\end{proof}

\begin{theorem}
$\mathrm{dim} (V_h) = N$.
\end{theorem}
\begin{proof}
For any $v_h\in V_h$, we observe that on each subinterval $I_j$ for $j=1,\cdots,N+1$, $v|_{I_j}$ is a linear polynomial and thus uniquely determined by $2$ parameters, known as the \textit{degree of freedom}. 
Since there are $N+1$ subintervals, this initially gives a total of $2(N+1)$ degrees of freedom. However, imposing $N$ continuity conditions at the subinterval boundaries and $2$ boundary conditions reduces the count by $N+2$, leaving $2(N+1)-N-2 = N$ degrees of freedom. Consequently, the dimension of the space is $N$.
\end{proof}

\begin{remark}
Why nodal basis functions?
\end{remark}

Let us introduce the linear basis function $\phi_j(x)$ for $1\leqslant j \leqslant N$, which satisfies the properties
\[
\phi_j\left(x_i\right)=\left\{\begin{array}{l}
1,\ \text { if } i=j, \\
0,\ \text { if } i \neq j.
\end{array}\right.
\]
Then $\phi_j(x) \in V_h$ and $\{\phi_1(x),\cdots,\phi_N(x)\}$ is linear independent and thus, by dimension argument, constitutes a basis for $V_h$, i.e., $V_h = \mathrm{span} \{\phi_1,\cdots,\phi_N \}$. Consequently, $\forall v_h \in V_h$, there is an unique representation
\[
v_h(x)=\sum_{j=1}^N v_j \phi_j(x), \quad x \in[0,1],
\]
where $v_j = v_h(x_j)$. More specifically, $\phi_j$ is given by
\begin{equation}
\phi_j(x)=\left\{\begin{array}{cc}
\frac{x-x_{j-1}}{h_j}, & \text { if } x \in\left[x_{j-1}, x_j\right], \\
\frac{x_{j+1}-x}{h_{j+1}}, & \text { if } x \in\left[x_j, x_{j+1}\right], \\
0, & \text { elsewhere }.
\end{array}\right.
\end{equation}

With the constructed piecewise linear space $V_h = \mathrm{span}\{\phi_1,\cdots,\phi_N\}$, we set the solution $u_h$ of \eqref{eq:1d-gm} as
\[
u_h(x)=\sum_{j=1}^N u_j \phi_j(x), \quad u_j=u_h\left(x_j\right).
\]
Substituting $u_h$ in \eqref{eq:1d-gm} and choosing $v = \phi_i(x)$ in \eqref{eq:1d-gm} for each $i=1,\cdots,N$, we obtain
\[
\sum_{j=1}^N\left(\frac{d \phi_j}{d x}, \frac{d \phi_i}{d x}\right) u_j=\left(f, \phi_i\right) \quad 1 \leq i \leq N,
\]
which is a linear system of $N$ equations with $N$ unknowns $u_j$:
\[
\mathbf{A} \mathbf{u} = \mathbf{F},
\]
where $\mathbf{u} = [u_1,\cdots,u_N]^{\mathrm{T}}$, $\mathbf{F} = [F_1,\cdots,F_N]^\mathrm{T}$ with elements $F_i = (f,\phi_i)$, and $\mathbf{A} = (a_{i, j})$ is an $N\times N$ matrix with elements $a_{i,j} = (\frac{d \phi_j}{dx}, \frac{d\phi_i}{dx})$.

The matrix $\mathbf{A}$ is called the \textit{stiffness matrix} and $\mathbf{F}$ the \textit{load vector}. We can explicitly calculate the elements in $\mathbf{A}$:
\[
\begin{aligned}
& a_{j,j} = \left(\frac{d \phi_j}{d x}, \frac{d \phi_j}{d x}\right)=\int_{x_{j-1}}^{x_j} \frac{1}{h_j^2} d x+\int_{x_j}^{x_{j+1}} \frac{1}{h_{j+1}^2} d x=\frac{1}{h_j}+\frac{1}{h_{j+1}}, \quad 1 \leq j \leq N, \\
& a_{j-1,j} = \left(\frac{d \phi_j}{d x}, \frac{d \phi_{j-1}}{d x}\right)=\int_{x_{j-1}}^{x_j} \frac{-1}{h_j^2} d x=-\frac{1}{h_j}, \quad 2 \leq j \leq N, \\
& a_{j,j-1}  =\left(\frac{d \phi_{j-1}}{d x}, \frac{d \phi_j}{d x}\right) = a_{j-1,j} = -\frac{1}{h_j}, \quad 2 \leq j \leq N,\\
& a_{i,j} = \left(\frac{d \phi_j}{d x}, \frac{d \phi_i}{d x}\right)=0, \quad \text {if} \quad|j-i|>1 .
\end{aligned}
\]
Thus the matrix $\mathbf{A}$ is tri-diagonal. Let $\mathbf{v} = [v_1,\cdots,v_N]^\mathrm{T}$, and we note that
\[
\mathbf{v}^\mathrm{T} \mathbf{A} \mathbf{v}
= \sum_{i,j=1}^N a_{i,j} v_i v_j = \sum_{i,j=1}^N v_j \left (\frac{d\phi_j}{dx}, \frac{d\phi_i}{dx}  \right ) v_i = \left ( \sum_{j=1}^N v_j \frac{d\phi_j}{dx}, \sum_{i=1}^N v_i \frac{d \phi_i}{dx} \right) = \left (\frac{dv_h}{dx}, \frac{dv_h}{dx} \right ) \geqslant 0,
\]
where we denote $v_h(x) = \sum_{j=1}^N v_j \phi_j(x)$. Thus the equality holds if and only if $\frac{dv_h}{dx}\equiv 0$, which is equivalent to $v_h(x)$ is constant, and by $v_h(0) = 0$ we have $v_h(x)\equiv 0$, or $\mathbf{v} = \mathbf{0}$. Therefore $\mathbf{A}$ is positive definite, which guarantees the linear system has a unique solution. 


\begin{itemize}
\item $\mathbf{A}$ is symmetric: $a_{i,j} = a_{j,i}$,
\item $\mathbf{A}$ is sparse: $a_{i,j} = 0$ for $|i-j|>1$,
\item $\mathbf{A}$ is positive definite.
\end{itemize}

In a particular case: $h_j = h = \frac{1}{N+1}$, we have
\[
\mathbf{A} = \frac{1}{h} 
\begin{bmatrix}
2 &  -1 &   &  &  &  &  \\
-1 & 2 & -1 &  &  &  &  \\
  & -1 & 2 & -1 &  &  &  \\
 &  & \ddots&\ddots  & \ddots  & &  \\
  &  &  & \ddots  &\ddots & \ddots &  \\
 &  &  & & -1 & 2 &-1 \\
 &  &  & &  & -1 &2 \\
\end{bmatrix}_{N\times N}
\]

\begin{theorem}
Eigenvalue of $\mathbf{A}$ is
\end{theorem}


\subsection{Error Estimate For $P1-$FEM}
Let $u \in C(\bar{I})$. We denote $u_I$ the interpolation of $u$ into $V_h$ at nodes $\{x_j\}_{j=0}^N$, i.e., $u_I \in V_h$ and 
\[
u_I(x_j) = u(x_j),\quad j=0,\cdots,N.
\]
It is evident that $u_I(x) = \sum_{j=0}^N u(x_j) \phi_j(x)$.

\subsubsection{Interpolation Error bounded by $L^\infty-$norm}

\begin{theorem}
\[
\|u-u_I\|_\infty \leqslant \frac{h^2}{8} \max_{x\in \bar{I}} |u^{\prime \prime}(x)|.
\]
\end{theorem}




\section{$P2-$ FEM}


\section{Implementation in General Framework}

\subsection{Target Problem}
Let $L$ be the second order linear operator defined by
\[
L w : = - (a u^\prime)^\prime
\]
\subsection{Finite Element Spaces}

\subsection{Finite Element Discretization}

\subsection{Boundary Treatment}

\subsection{Finite Element Method}


\newpage
\begin{thebibliography}{99}
\bibitem[Evans (2010)]{evans2010} Evans L C. Partial differential equations[M]. American Mathematical Society, Second Edition, 2010.


\end{thebibliography}


\end{document}

