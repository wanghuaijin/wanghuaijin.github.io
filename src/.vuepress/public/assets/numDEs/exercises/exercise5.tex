
% This LaTeX was auto-generated from MATLAB code.
% To make changes, update the MATLAB code and republish this document.

\documentclass[11pt]{article}
\usepackage[utf8]{inputenc}
\usepackage[T1]{fontenc}
\usepackage{amsthm}
\usepackage{enumitem}
\usepackage{amssymb}
\usepackage{amsmath}
\usepackage{amsfonts}
\usepackage[version=4]{mhchem}
\usepackage{stmaryrd}
\usepackage{mathrsfs}
\usepackage{bm}
\usepackage{graphicx}
\usepackage[export]{adjustbox}
\graphicspath{ {./images/} }
\usepackage{algorithm}
\usepackage{algorithmic}
\usepackage{makecell}  % 表格换行


\usepackage{hyperref}
\hypersetup{
    colorlinks=true,     % 启用颜色链接
    linkcolor=blue,     % 内部链接的颜色
    citecolor=blue,      % 引用文献的颜色
    urlcolor=blue,       % URL链接的颜色
    linktoc=red,      % 不影响目录链接颜色
}

\usepackage[a4paper, top=1in, bottom=1in, left=1in, right=1in]{geometry}

\title{
{\bf \huge Notes on M\"untz-Jackson Theorem}
%{\bf \large For M\"untz systems on [0,1]}
}
\author{Huaijin Wang}
\date{December 10, 2024}


\begin{document}


\newtheorem{definition}{Definition}[section]
\newtheorem{property}{Property}[section]
\newtheorem{lemma}{Lemma}[section]
\newtheorem{theorem}{Theorem}[section]
\newtheorem{corollary}{Corollary}[section]
\newtheorem{remark}{Remark}[section]
\newtheorem{example}{Example}[]
\newtheorem{notation}{Notation Declaration}[]
\newtheorem{question}{Question}[]
\newtheorem{exercise}{Exercise}[section]

%\maketitle




\newpage



\setcounter{section}{2}
\setcounter{exercise}{14}
\begin{exercise}
Consider the elliptic problem
\[
\begin{aligned}
-u&_{xx}  =f, \quad \forall x\in (a,b),\\
& u(a) = 0, \ u^\prime(b) = \beta,
\end{aligned}
\]
and its finite difference schema
\[
\begin{aligned}
 - \frac{u_{i+1} - 2u_i + u_{i-1}}{h^2} = f_i,& \quad \forall i=1,\cdots,N-1,\\
 u_0 = 0,& \\
 \frac{u_{N}- u_{N-1}}{h} = \beta, & \\
\end{aligned}
\]
in an uniform mesh $\{x_i\}_{i=0}^{N}$, $x_i = a+ih$, $h = (b-a)/N$. \\
1) Derive an estimate for the truncation errors:
\[
R_i^{(1)} = L_h [u(x_i)] - [L u] (x_i) \text{  for  } i=1,\cdots,N-1, \ R^{(2)} = \frac{u(x_{N}) - u(x_{N-1})}{h} - u^\prime(x_{N}).
\]
2) Rewrite the discrete problem under matrix form. \\
3) Establish an a priori estimate for $\|u_h\|_1$. \\
4) Derive an error estimate for $\|e_h\|_1$, where $e_i = u(x_i) - u_i$.
\end{exercise}


\newpage
\section*{Appendix: Notations for Discrete Representation}
 Let $I = [a,b]$. We define the discrete grid points as
\[
a=x_0<x_1<\cdots<x_N = b.
\]
We introduce the following sets:
\[
I_h = \{x_1,\cdots,x_{N-1}\}, \ \bar{I}_h = \{x_0,x_1,\cdots, x_N\}, \ I_h^+ = \{x_1,\cdots,x_N\}.
\]
The grid spacing is defined as
\[
h_i = x_{i}- x_{i-1}, \quad i=1,\cdots,N.
\]
Additionally, we define the averaged grid spacing:
\[
\begin{aligned}
& \bar{h}_i = \frac{1}{2} (h_i+h_{i+1}), \ i=1,\cdots,N-1,\\
& \bar{h}_0  = \frac{1}{2} h_1, \quad \bar{h}_N = \frac{1}{2} h_N.
\end{aligned}
\]
A discrete function defined on $\bar{I}_h$ is denoted as 
\[
v_h = \{v_0,v_1,\cdots, v_N \}.
\]
We define the following difference operators:
\[
\begin{aligned}
& (v_i)_{\bar{x}} := v_{i,\bar{x}} : = \frac{v_i-v_{i-1}}{h_i}, \ i =1,\cdots,N, \\
& (v_i)_x := v_{i, x} := \frac{v_{i+1} - v_i}{h_{i+1}},\ i=0,\cdots,N-1, \\
&  (v_i)_{\hat{x}} := v_{i, \hat{x}} := \frac{v_{i+1} - v_i}{\bar{h}_{i}},\ i=0,\cdots,N-1.
\end{aligned}
\]
The discrete inner products are given by
\begin{equation}
(u_h, v_h)_{I_h} = \sum_{i=1}^{N-1} u_i v_i \bar{h}_i, \
(u_h, v_h)_{\bar{I}_h} = \sum_{i=0}^{N} u_i v_i \bar{h}_i, \
(u_h, v_h)_{I^+_h} = \sum_{i=1}^{N} u_i v_i h_i.
\label{eq:d-ip}
\end{equation}
We define the discrete norms as follows:
\begin{equation}
\begin{aligned}
& \|v_h\|_c := \max_{\bar{I}_h} |v_i|,\ \|v_h\|_0 := (v_h,v_h)_{\bar{I}_h}^{1/2}, \\
& |v_h|_1 := ((v_h)_{\bar{x}}, (v_h)_{\bar{x}})_{I_h^+}^{1/2}, \ \|v_h\|_1^2 = \|v_h\|_0^2 + |v_h|_1^2.
\end{aligned}
\label{eq:d-in}
\end{equation}
The discrete integral by parts:
\begin{equation}
\sum_{i=m+1}^n v_i (w_i)_{\bar{x}} h_i = - \sum_{i=m}^{n-1} (v_i)_x w_i h_{i+1} + v_n w_n - v_m w_m,\ \text{for some} \ 0\leqslant m < n \leqslant N.
\label{eq:d-ibp}
\end{equation}
The discrete Green formula:
\begin{equation}
\sum_{i=m+1}^{n-1} \left( (u_i)_{\bar{x}} \right )_{\hat{x}} v_i \bar{h}_i = - \sum_{i=m+1}^n (u_i)_{\bar{x}} (v_i)_{\bar{x}} h_i + (u_n)_{\bar{x}} v_n - (u_m)_x v_m,\ \text{for some} \ 0\leqslant m < n \leqslant N.
\label{eq:d-gf}
\end{equation}
The discrete Cauchy-Schwarz inequality states that
\begin{equation}
|(u_h, v_h)_{\bar{I}_h}| \leqslant (u_h, u_h)_{\bar{I}_h}^{1/2} (v_h, v_h)_{\bar{I}_h}^{1/2}.
\label{eq:d-csi}
\end{equation}
If $v_0 = 0$ (or $v_N=0$ or $v_0=v_N=0$), the discrete Poincar\'e inequality holds:
\begin{equation}
\|v_h\|_c \leqslant C |v_h|_1, \quad \|v_h\|_0 \leqslant C |v_h|_1,
\label{eq:d-pi}
\end{equation}
where $C$ is a constant depending only on $a$ and $b$.


\end{document}

