
% This LaTeX was auto-generated from MATLAB code.
% To make changes, update the MATLAB code and republish this document.

\documentclass[11pt]{article}
\usepackage[utf8]{inputenc}
\usepackage[T1]{fontenc}
\usepackage{amsthm}
\usepackage{enumitem}
\usepackage{amssymb}
\usepackage{amsmath}
\usepackage{amsfonts}
\usepackage[version=4]{mhchem}
\usepackage{stmaryrd}
\usepackage{mathrsfs}
\usepackage{bm}
\usepackage{graphicx}
\usepackage[export]{adjustbox}
\graphicspath{ {./images/} }
\usepackage{algorithm}
\usepackage{algorithmic}
\usepackage{makecell}  % 表格换行


\usepackage{hyperref}
\hypersetup{
    colorlinks=true,     % 启用颜色链接
    linkcolor=blue,     % 内部链接的颜色
    citecolor=blue,      % 引用文献的颜色
    urlcolor=blue,       % URL链接的颜色
    linktoc=red,      % 不影响目录链接颜色
}

\usepackage[a4paper, top=1in, bottom=1in, left=1in, right=1in]{geometry}

\title{
{\bf \huge Notes on M\"untz-Jackson Theorem}
%{\bf \large For M\"untz systems on [0,1]}
}
\author{Huaijin Wang}
\date{December 10, 2024}


\begin{document}


\newtheorem{definition}{Definition}[section]
\newtheorem{property}{Property}[section]
\newtheorem{lemma}{Lemma}[section]
\newtheorem{theorem}{Theorem}[section]
\newtheorem{corollary}{Corollary}[section]
\newtheorem{remark}{Remark}[section]
\newtheorem{example}{Example}[]
\newtheorem{notation}{Notation Declaration}[]
\newtheorem{question}{Question}[]
\newtheorem{exercise}{Exercise}[]

%\maketitle




\newpage

\setcounter{section}{1}
\setcounter{exercise}{0}
\begin{exercise}
Let $\{x_n\}_{n=0}^{N+1}$ be a grid in the interval $\Lambda = (0,1)$, i.e., $0=x_0<x_1 <x_2<\cdots < x_N < x_{N+1} =1$. Let $I_n = (x_{n-1},x_n)$, $h_n=x_n-x_{n-1}$, and $h=\max_{1\leqslant n \leqslant N+1} h_n$. Prove
\[
\{ v\in C^0 (\Lambda): v|_{I_n} \in H^1(I_n), n = 1,\cdots,N+1\} \subset H^1(\Lambda).
\]
\end{exercise}
\begin{proof}
For any $v\in \{ v\in C^0 (\Lambda): v|_{I_n} \in H^1(I_n), n = 1,\cdots,N+1\}$, it is clear that $v\in L^2(\Lambda)$ because continuity implies square integrability on the bounded domain $\Lambda$. 
It remains to show that the weak derivative of $v$ also belongs to $L^2(\Lambda)$. Since $v|_{I_n}\in H^1(I_n)$, we define a piecewise derivative  by
\[
g|_{I_n} (x) = (v|_{I_n})^\prime (x), \quad x\in I_n, \ n=1,\cdots,N+1.
\]
Obviously, $g\in L^2(\Lambda)$, as each piece $(v|_{I_n})^\prime\in L^2(I_n)$ and the intervals $I_n$ are disjoint and cover $\Lambda$. We claim that $g$ is the derivative of $v$. Indeed, for any test function $ \phi(x) \in C_0^\infty(\Lambda)$, we have
\[
\begin{aligned}
\int_0^1 g(x) \phi(x)  \mathrm{d} x & = \sum_{n=1}^{N+1} \int_{I_n} g |_{I_n} (x) \phi (x) \mathrm{d} x = \sum_{n=1}^{N+1} \int_{I_n} (v|_{I_n})^\prime (x) \phi(x) \mathrm{d} x \\
& = \sum_{n=1}^{N+1} [v(x)\phi(x)] \big |_{x_{n-1}}^{x_n} - \sum_{n=1}^{N+1} \int_{I_n} (v|_{I_n}) (x) \phi^\prime (x) \mathrm{d} x \\
& = \sum_{n=1}^{N+1} \left(v(x_n^-)\phi(x_n^-) - v(x_{n-1}^+) \phi(x_{n-1}^+) \right)
-\sum_{n=1}^{N+1} \int_{I_n} (v|_{I_n}) (x) \phi^\prime (x) \mathrm{d} x.
\end{aligned}
\]
Due to the continuity of $v$ across element interfaces, we have $v(x_n^-) = v(x_n^+)$ for $n=1,\cdots,N$, and since $\phi\in C_0^\infty(\Lambda)$ we have  $\phi(x_0)=\phi(x_{N+1})=0$. Hence, the sum of boundary terms cancels out, yielding
\[
\int_0^1 g(x) \phi(x) \mathrm{d} x=-\int_0^1 v(x) \phi^{\prime}(x) \mathrm{d} x,
\]
which confirms that $g$ is the weak derivative of $v$. Therefore, $v\in H^1(\Lambda)$.
\end{proof}


\end{document}

