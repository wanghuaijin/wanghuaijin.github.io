
% This LaTeX was auto-generated from MATLAB code.
% To make changes, update the MATLAB code and republish this document.

\documentclass[11pt]{article}
\usepackage[utf8]{inputenc}
\usepackage[T1]{fontenc}
\usepackage{amsthm}
\usepackage{enumitem}
\usepackage{amssymb}
\usepackage{amsmath}
\usepackage{amsfonts}
\usepackage[version=4]{mhchem}
\usepackage{stmaryrd}
\usepackage{mathrsfs}
\usepackage{bm}
\usepackage{graphicx}
\usepackage[export]{adjustbox}
\graphicspath{ {./images/} }
\usepackage{algorithm}
\usepackage{algorithmic}
\usepackage{makecell}  % 表格换行


\usepackage{hyperref}
\hypersetup{
    colorlinks=true,     % 启用颜色链接
    linkcolor=blue,     % 内部链接的颜色
    citecolor=blue,      % 引用文献的颜色
    urlcolor=blue,       % URL链接的颜色
    linktoc=red,      % 不影响目录链接颜色
}

\usepackage[a4paper, top=1in, bottom=1in, left=1in, right=1in]{geometry}

\title{
{\bf \huge Notes on M\"untz-Jackson Theorem}
%{\bf \large For M\"untz systems on [0,1]}
}
\author{Huaijin Wang}
\date{December 10, 2024}


\begin{document}


\newtheorem{definition}{Definition}[section]
\newtheorem{property}{Property}[section]
\newtheorem{lemma}{Lemma}[section]
\newtheorem{theorem}{Theorem}[section]
\newtheorem{corollary}{Corollary}[section]
\newtheorem{remark}{Remark}[section]
\newtheorem{example}{Example}[]
\newtheorem{notation}{Notation Declaration}[]
\newtheorem{question}{Question}[]
\newtheorem{exercise}{Exercise}[section]

%\maketitle




\newpage



\setcounter{section}{2}
\setcounter{exercise}{13}
\begin{exercise}
Consider the elliptic problem
\[
\begin{aligned}
-u&_{xx} + u_x + u =f, \quad \forall x\in (a,b),\\
& u(a) = u(b) = 0,
\end{aligned}
\]
and its finite difference schema 
\begin{equation}
\begin{aligned}
  -\frac{u_{i+1} - 2u_i + u_{i-1}}{h^2} + \frac{u_{i+1} - u_{i-1}}{2h} + u_i = f_i, & \quad \forall i=1,\cdots,N-1, \\
 u_0 = u_N = 0, & \\ 
 \end{aligned}
 \label{eq:2-14}
\end{equation}
in an uniform mesh $\{x_i \}_{i=0}^N$, $x_i = a+ih$, $h = (b-a)/N$. \\
1) Derive an estimate for the truncation error; \\
2) Establish an a priori estimate for $\|u_h\|_1$; \\
3) Prove the existence and uniqueness of the solution of the finite difference schema; \\
4) Derive an  error estimate for $\|e_h\|_1$, where $e_i = u(x_i) - u_i$.
\end{exercise}
\begin{proof}[Solution]
1). Let the operator $L u = - u_{xx} + u_x + u$ and the discrete operator $L_h$ on $\{u_i\}_{i=1}^{N-1}$ as
\[
L_h u_i =  -\frac{u_{i+1} - 2u_i + u_{i-1}}{h^2} + \frac{u_{i+1} - u_{i-1}}{2h} + u_i.
\]
Then truncation error $R_i = L_h [u(x_i)] - [L u] (x_i)$. By the Tylor development
\[
u(x_{i+1}) = u(x_i) + h u^\prime(x_i) + \frac{h^2}{2} u^{\prime \prime} (x_i) + \frac{h^3}{3!} u^{(3)} (x_i) + \frac{h^4}{4!} u^{(4)} (\xi_i), \ \text{for some} \ \xi_i \in (x_i, x_{i+1}),
\]
\[
u(x_{i-1}) = u(x_i) - h u^\prime(x_i) + \frac{h^2}{2} u^{\prime \prime} (x_i) - \frac{h^3}{3!} u^{(3)} (x_i) + \frac{h^4}{4!} u^{(4)} (\eta_i), \ \text{for some} \ \eta_i \in (x_{i-1},x_{i}),
\]
we obtain that $R_i = O(h^2)$ as $h \to 0$ for $i=1,\cdots,N-1$. \\
2). Note that $L_h u_i=-\left(\left(u_i\right)_{\bar{x}}\right)_{\hat{x}}+\frac{1}{2}\left(\left(u_i\right)_{\bar{x}}+\left(u_i\right)_x\right)+u_i$, then multiplying both sides of the finite difference schema $L_h u_i = f_i$ by $u_i h_i$ yields
\[
-\left(\left(u_i\right)_{\bar{x}}\right)_{\hat{x}} u_i h_i + \frac{1}{2} \left(\left(u_i\right)_{\bar{x}}+\left(u_i\right)_x\right) u_i h + u_i^2 h = f_i u_i h_i,\quad \forall i=1,\cdots,N-1.
\]
Summing in $i$ gives
\[
- \left(\left(\left(u_h\right)_{\bar{x}}\right)_{\hat{x}}, u_h \right )_{I_h} + \frac{1}{2} \left( (u_h)_{\bar{x}}, u_h \right)_{I_h} + \frac{1}{2} \left( (u_h)_{x}, u_h \right)_{I_h} + \left( u_h , u_h \right)_{I_h}= (f_h, u_h)_{I_h}.
\]
In virtue of discrete integral by parts \eqref{eq:d-ibp}, discrete Green formula \eqref{eq:d-gf} and the fact that $u_0 = u_N = 0$, we have 
\[
- \left (\left(\left(u_h\right)_{\bar{x}}\right )_{\hat{x}}, u_h \right)_{I_h}  =  \left( (u_h)_{\bar{x}}, (u_h)_{\bar{x}} \right)_{I_h^+}, \quad \left( (u_h)_{\bar{x}}, u_h \right)_{I_h} = - \left( (u_h)_x, u_h \right)_{I_h} .
\]
Thus 
\[
\left( (u_h)_{\bar{x}}, (u_h)_{\bar{x}} \right)_{I_h^+} + \left (u_h,u_h \right )_{I_h} = \left (f_h, u_h \right)_{I_h}.
\]
Using the fact that $u_0=u_N=0$, it is equivalent to
\[
\left( (u_h)_{\bar{x}}, (u_h)_{\bar{x}} \right)_{I_h^+} + \left (u_h,u_h \right )_{\bar{I}_h} = \left (f_h, u_h \right)_{\bar{I}_h}.
\]
By the definition of the discrete inner norm \eqref{eq:d-in}, the left-hand side of above formula is $\|u_h\|_1^2$.
By the discrete Cauchy-Schwarz inequality \eqref{eq:d-csi}, and the discrete Poincar\'e inequality \eqref{eq:d-pi}: $\|u_h\|_0 \leqslant C |u_h|_1 \leqslant C \|u_h\|_1$, we have
\[
\|u_h\|_1^2 \leqslant \|f_h\|_0 \|u_h\|_0 \leqslant C\|f_h\|_0 \| u_h\|_1 \ \Longrightarrow \ \|u_h\|_1 \leqslant C \|f_h\|_0.
\]
3). The finite difference schema is equivalent to solve the linear system:
\[
\mathbf{D} \mathbf{u} = \mathbf{f},
\]
where $\mathbf{u} = [u_1,\cdots, u_{N-1}]^{\mathrm{T}}$, $\mathbf{f} = [f_1,\cdots,f_{N-1}]^{\mathrm{T}}$ and
\[
\mathbf{D}=\left [ \begin{array}{ccccc}
1+\frac{2}{h^2} & -\frac{1}{h^2}+\frac{1}{2 h} & & & \\
-\frac{1}{h^2}-\frac{1}{2 h} & 1+\frac{2}{h^2} & -\frac{1}{h^2}+\frac{1}{2 h} & & \\
& \ddots & \ddots & \ddots & \\
& & -\frac{1}{h^2}-\frac{1}{2 h} & 1+\frac{2}{h^2} & -\frac{1}{h^2}+\frac{1}{2 h} \\
& & & -\frac{1}{h^2}-\frac{1}{2 h} & 1+\frac{2}{h^2}
\end{array}\right].
\]
Note that $\mathbf{D}$ is strictly diagonally dominant, i.e.,
\[
\sum_{j=1, j\neq i}^{N-1} |D_{ij}| < |D_{ii}|,\quad i = 1,\cdots,N-1.
\]
Then $\mathbf{D}$ is nonsingular, which leads to the existence and uniqueness of the solution of the finite difference schema. \\
4). It is obvious that 
\[
\left \{
\begin{aligned}
& L_h e_i = R_i, \ i=1,\cdots,N-1, \\
& e_0 = e_N = 0.
\end{aligned}
\right.
\]
By 1) and 2) we have $\|e_h\|_1 \leqslant C\|R_h\|_0 = O(h^2)$ as $h\to 0$.
\end{proof}



\newpage
\section*{Appendix: Notations for Discrete Representation}
 Let $I = [a,b]$. We define the discrete grid points as
\[
a=x_0<x_1<\cdots<x_N = b.
\]
We introduce the following sets:
\[
I_h = \{x_1,\cdots,x_{N-1}\}, \ \bar{I}_h = \{x_0,x_1,\cdots, x_N\}, \ I_h^+ = \{x_1,\cdots,x_N\}.
\]
The grid spacing is defined as
\[
h_i = x_{i}- x_{i-1}, \quad i=1,\cdots,N.
\]
Additionally, we define the averaged grid spacing:
\[
\begin{aligned}
& \bar{h}_i = \frac{1}{2} (h_i+h_{i+1}), \ i=1,\cdots,N-1,\\
& \bar{h}_0  = \frac{1}{2} h_1, \quad \bar{h}_N = \frac{1}{2} h_N.
\end{aligned}
\]
A discrete function defined on $\bar{I}_h$ is denoted as 
\[
v_h = \{v_0,v_1,\cdots, v_N \}.
\]
We define the following difference operators:
\[
\begin{aligned}
& (v_i)_{\bar{x}} := v_{i,\bar{x}} : = \frac{v_i-v_{i-1}}{h_i}, \ i =1,\cdots,N, \\
& (v_i)_x := v_{i, x} := \frac{v_{i+1} - v_i}{h_{i+1}},\ i=0,\cdots,N-1, \\
&  (v_i)_{\hat{x}} := v_{i, \hat{x}} := \frac{v_{i+1} - v_i}{\bar{h}_{i}},\ i=0,\cdots,N-1.
\end{aligned}
\]
The discrete inner products are given by
\begin{equation}
(u_h, v_h)_{I_h} = \sum_{i=1}^{N-1} u_i v_i \bar{h}_i, \
(u_h, v_h)_{\bar{I}_h} = \sum_{i=0}^{N} u_i v_i \bar{h}_i, \
(u_h, v_h)_{I^+_h} = \sum_{i=1}^{N} u_i v_i h_i.
\label{eq:d-ip}
\end{equation}
We define the discrete norms as follows:
\begin{equation}
\begin{aligned}
& \|v_h\|_c := \max_{\bar{I}_h} |v_i|,\ \|v_h\|_0 := (v_h,v_h)_{\bar{I}_h}^{1/2}, \\
& |v_h|_1 := ((v_h)_{\bar{x}}, (v_h)_{\bar{x}})_{I_h^+}^{1/2}, \ \|v_h\|_1^2 = \|v_h\|_0^2 + |v_h|_1^2.
\end{aligned}
\label{eq:d-in}
\end{equation}
The discrete integral by parts:
\begin{equation}
\sum_{i=m+1}^n v_i (w_i)_{\bar{x}} h_i = - \sum_{i=m}^{n-1} (v_i)_x w_i h_{i+1} + v_n w_n - v_m w_m,\ \text{for some} \ 0\leqslant m < n \leqslant N.
\label{eq:d-ibp}
\end{equation}
The discrete Green formula:
\begin{equation}
\sum_{i=m+1}^{n-1} \left( (u_i)_{\bar{x}} \right )_{\hat{x}} v_i \bar{h}_i = - \sum_{i=m+1}^n (u_i)_{\bar{x}} (v_i)_{\bar{x}} h_i + (u_n)_{\bar{x}} v_n - (u_m)_x v_m,\ \text{for some} \ 0\leqslant m < n \leqslant N.
\label{eq:d-gf}
\end{equation}
The discrete Cauchy-Schwarz inequality states that
\begin{equation}
|(u_h, v_h)_{\bar{I}_h}| \leqslant (u_h, u_h)_{\bar{I}_h}^{1/2} (v_h, v_h)_{\bar{I}_h}^{1/2}.
\label{eq:d-csi}
\end{equation}
If $v_0 = 0$ (or $v_N=0$ or $v_0=v_N=0$), the discrete Poincar\'e inequality holds:
\begin{equation}
\|v_h\|_c \leqslant C |v_h|_1, \quad \|v_h\|_0 \leqslant C |v_h|_1,
\label{eq:d-pi}
\end{equation}
where $C$ is a constant depending only on $a$ and $b$.


\end{document}

