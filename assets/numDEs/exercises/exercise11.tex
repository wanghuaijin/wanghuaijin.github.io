
% This LaTeX was auto-generated from MATLAB code.
% To make changes, update the MATLAB code and republish this document.

\documentclass[11pt]{article}
\usepackage[utf8]{inputenc}
\usepackage[T1]{fontenc}
\usepackage{amsthm}
\usepackage{enumitem}
\usepackage{amssymb}
\usepackage{amsmath}
\usepackage{amsfonts}
\usepackage[version=4]{mhchem}
\usepackage{stmaryrd}
\usepackage{mathrsfs}
\usepackage{bm}
\usepackage{graphicx}
\usepackage[export]{adjustbox}
\graphicspath{ {./images/} }
\usepackage{algorithm}
\usepackage{algorithmic}
\usepackage{makecell}  % 表格换行


\usepackage{hyperref}
\hypersetup{
    colorlinks=true,     % 启用颜色链接
    linkcolor=blue,     % 内部链接的颜色
    citecolor=blue,      % 引用文献的颜色
    urlcolor=blue,       % URL链接的颜色
    linktoc=red,      % 不影响目录链接颜色
}

\usepackage[a4paper, top=1in, bottom=1in, left=1in, right=1in]{geometry}

\title{
{\bf \huge Notes on M\"untz-Jackson Theorem}
%{\bf \large For M\"untz systems on [0,1]}
}
\author{Huaijin Wang}
\date{December 10, 2024}


\begin{document}


\newtheorem{definition}{Definition}[section]
\newtheorem{property}{Property}[section]
\newtheorem{lemma}{Lemma}[section]
\newtheorem{theorem}{Theorem}[section]
\newtheorem{corollary}{Corollary}[section]
\newtheorem{remark}{Remark}[section]
\newtheorem{example}{Example}[]
\newtheorem{notation}{Notation Declaration}[]
\newtheorem{question}{Question}[]
\newtheorem{exercise}{Exercise}[section]

%\maketitle




\newpage

\setcounter{section}{2}
\setcounter{exercise}{3}
\begin{exercise}
We consider the problem
\begin{equation}
\left \{
\begin{aligned}
& -(\alpha u^\prime)^\prime (x) + (\beta u^\prime) (x) + (\gamma u) (x) = f(x),\quad x\in (0,1), \\
& u(0) = u(1) = 0,
\end{aligned}
\right .
\tag{11}
\label{eq:11}
\end{equation}
where $\alpha,\beta$, and $\gamma$ are continuous functions on $[0,1]$ with $\alpha(x) \geqslant \alpha_0 >0$ for all $x\in [0,1]$.
~\\
1) Give the weak form of the problem \eqref{eq:11}.
~\\
2) Prove the weak problem admits a unique solution under the following assumption
\begin{enumerate}[label=\alph*.]
\item $\beta(x)=0$, $\gamma \geqslant 0$ for all $x\in [0,1]$;
\item $-\frac{1}{2} \beta^\prime + \gamma \geqslant 0$ for all $x\in [0,1]$; 
\item see [Brezis p. 224].
\end{enumerate}
3) Propose a $P1-$FEM for the numerical solution of \eqref{eq:11}.
~\\
4) Carry out an error analysis.
\end{exercise}
\begin{proof}
~\\
1). Let $I = (0,1)$ and $V = H_0^1(I)$. The weak form reads
\begin{equation}
\left \{
\begin{aligned}
& \text{Find}\ u\in V \ \text{such that} \\
& a(u,v) = \mathcal{F}(v), \ \forall v\in V,
\end{aligned}
\right .
\label{eq:1-1}
\end{equation}
where the bilinear form $a(u,v) = (\alpha u^\prime, v^\prime) + (\beta u^\prime, v^\prime) + (\gamma u,v)$ and $\mathcal{F}(v) = (f,v)$.

\noindent 2). It is clear that $a(\cdot,\cdot)$ is a bilinear form, and $\mathcal{F}$ is a continuous functional from $V$ to $\mathbb{R}$. By Lax-Milgram Lemma, it remains to show that $a(\cdot,\cdot)$ is continuous and coercive. Continuity is clear for all cases since the coefficients are continuous over $\bar{I}$, i.e.,
\[
\begin{aligned}
|a(u,v)| & \leqslant \|\alpha\|_\infty \|u^\prime\|_0 \|v^\prime\|_0 + \|\beta\|_\infty \|u^\prime\|_0 \|v\|_0 + \|\gamma\|_\infty \|u\|_0 \|v\|_0 \\
& \leqslant (\|\alpha\|_\infty + \|\beta\|_\infty + \|\gamma\|_\infty ) \|u\|_1 \|v\|_1,\quad \forall u,v\in V.
\end{aligned}
\]
For the coercivity, we consider case by case:
\begin{enumerate}[label=\alph*.]
\item By Poincar\'e inequality on $H_0^1(I)$, there exists $C>0$ such that
\[
a(v,v) = (\alpha v^\prime, v^\prime) + (\gamma v, v) \geqslant \alpha_0\|v^\prime\|_0^2 \geqslant C \|v\|_1^2,\quad \forall v\in V.
\]
\item Note that for any $v\in V$, we have $(\beta v^\prime, v)=(\beta/2, (v^2)^\prime) = (-\beta^\prime/2, v^2)$. Thus by Poincar\'e inequality on $H_0^1(I)$, we have
\[
\begin{aligned}
a(v,v) & = (\alpha v^\prime, v^\prime) + (\beta v^\prime, v) + (\gamma v, v)
= (\alpha v^\prime, v^\prime) + ( (-\beta^\prime/2 + \gamma) v,v) \\
& \geqslant \alpha_0 \|v^\prime\|_0^2 \geqslant C\|v\|_1^2,\quad \forall v\in V.
\end{aligned}
\]
\item Omitted.
\end{enumerate}

\noindent 3). Let $0=x_0<x_1<\cdots<x_N<x_{N+1}=1$ be a grid on $I=(0,1)$. The space of piecewise linear polynomials is denoted by $X_h^1 = \mathrm{span} \{\varphi_0,\cdots,\varphi_{N+1} \}$. Then $V_h=V\cap X_h^1 = \mathrm{span}\{\varphi_1,\cdots,\varphi_N\}$. Let $u_h=\sum_{j=1}^N u_j \varphi_j(x)$ and
\[
\sum_{j=1}^N a(\varphi_j,\varphi_i) u_j = \mathcal{F}(\varphi_i),\quad i=1,\cdots,N.
\]

\noindent 4). Let $u_h$ be the solution of $P1-$FEM. Note that $a(u-u_h, v_h) = 0$, $\forall v_h\in V_h$. Thus
\[
\|u-u_h\|_1^2 \leqslant C a(u-u_h, u-u_h) = C a(u-u_h, u-v_h) \leqslant C\|u-u_h\|_1 \|u-v_h\|_1,
\]
for any $v_h\in V_h$. It leads to $\|u-u_h\|_1\leqslant \inf_{v_h\in V_h} \|u-v_h\|_1 \leqslant \|u-u_I\|_1$, where $u_I$ denotes the interpolation of $u$ into $V_h$. By the Poincar\'e inequality, we know that $C \|v\|_1 \leqslant \|v^\prime\|_0 \leqslant C \|v\|_1$, then
\[
\|u^\prime-u_h^\prime\|_0 \leqslant \|u^\prime - u_I^\prime\|_0 \leqslant C h \|u^{\prime\prime}\|_0.
\]
To obtain error estimate for $\|u-u_h\|_0$, we use duality argument. Consider the dual problem of \eqref{eq:1-1}: 
given $r\in L^2(I)$,
 \[
 \left\{
 \begin{aligned}
 &\text{Find}\ \varphi(r)\in V\ \text{such that} \\
 & a(v,\varphi(r)) = (r,v),\quad \forall v\in V.
 \end{aligned}
 \right.
 \]
The dual problem admits a unique solution $\varphi(r)$ since $a(\cdot,\cdot)$ is continuous and coercive.  If we suppose $\varphi(r)\in H^2(I)$, and there exists constant $C>0$ such that $\|\varphi^{\prime\prime}(r)\|_0 \leqslant C \|r\|_0$.

Thus we denote $\varphi_I(r)$ being the interpolation of $\varphi(r)$ into $V_h$ and obtain 
\[
\begin{aligned}
\|u-u_h\|_0 & = \sup_{r\in L^2(I), \ r\neq 0} \frac{(r,u-u_h)}{\|r\|_0} = 
\sup_{r\in L^2(I), \ r\neq 0}  \frac{a(u-u_h, \varphi(r))}{\|r\|_0} \\
& = \sup_{r\in L^2(I), \ r\neq 0}  \frac{a(u-u_h, \varphi(r) - \varphi_I (r))}{\|r\|_0} \\
& \leqslant C \sup_{r\in L^2(I), \ r\neq 0}  \frac{\|u-u_h\|_1 \|\varphi(r) - \varphi_I(r)\|_1 }{\|r\|_0} \\
 &  \leqslant  C \sup_{r\in L^2(I), \ r\neq 0}  \frac{\|u^\prime-u^\prime_h\|_0 \|\varphi^\prime(r) - \varphi^\prime_I(r)\|_0 }{\|r\|_0} \\
 & \leqslant C h\|u^\prime-u^\prime_h\|_0 \sup_{r\in L^2(I), \ r\neq 0} \frac{\|\varphi^{\prime\prime}(r)\|_0}{\|r\|_0}  \\
 & \leqslant C h\|u^\prime-u^\prime_h\|_0.
\end{aligned}
\]
\end{proof}




\setcounter{section}{3}
\setcounter{exercise}{0}
\begin{exercise}
Let $\Omega = (a,b)^2$, $f\in L^2(\Omega)$. Consider the Dirichlet elliptic problem
\[
\left\{
\begin{aligned}
&-\Delta u(\mathbf{x}) = f(\mathbf{x}),\quad \mathbf{x}\in \Omega,\\
&u\big |_\Gamma = 0.
\end{aligned}
\right.
\]
1. Prove the following Poincar\'e inequality holds: there exists a constant $c$, depending only on $a$ and $b$, such that
\[
\|v\|_1 \leqslant c |v|_1,\quad \forall v\in H_0^1(\Omega).
\]
2. Prove that the Dirichlet elliptic problem admits a unique weak solution in $H_0^1(\Omega)$, and the solution $u$ satisfies 
\[
\|u\|_1 \leqslant c\|f\|_0,
\]
where $c$ is a constant.
\end{exercise}
\begin{proof}
1. Note that for any $y\in (a,b)$,
\[
v(x,y) = \int_a^y \partial_y v(x,y) \mathrm{d} y,
\]
we have $|v(x,y)|\leqslant (b-a)^{1/2} \|\partial_y v\|_0$. Similarly we have $|v(x,y)|\leqslant (b-a)^{1/2} \|\partial_x v\|_0$. It leads to
\[
\|v\|_0 \leqslant \frac{(b-a)^{3/2}}{\sqrt{2}} |v|_1.
\]
Thus $\|v\|_1 = (\|v\|_0^2 + |v|_1^2)^{1/2} \leqslant c |v|_1$.

\vspace{1em}

\noindent 2. The variational form reads
\[
\left\{
\begin{aligned}
& \text{Find} \ u\in H_0^1(\Omega) \ \text{such that} \\
& a(u,v) = (f,v),\quad \forall v\in H_0^1(\Omega),
\end{aligned}
\right .
\]
where $a(u,v) = (\nabla u, \nabla v)$. It is clear that $a(\cdot,\cdot)$ is a continuous bilinear form. Its coercivity is guaranteed by Poincar\'e inequality, that is, 
\[
a(v,v) = |\nabla v|_1^2 \geqslant c\|v\|_1^2,\quad \forall v\in H_0^1(\Omega).
\]
Thus by Lax-Milgram lemma, there exists a unique weak solution $u$ such that
\[
\|u\|_1 \leqslant c \sup_{v\in H_0^1(\Omega), v\neq 0} \frac{(f,v)}{\|v\|_1} \leqslant c\|f\|_0.
\]
\end{proof}

\end{document}

