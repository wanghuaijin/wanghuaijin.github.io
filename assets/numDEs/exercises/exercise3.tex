
% This LaTeX was auto-generated from MATLAB code.
% To make changes, update the MATLAB code and republish this document.

\documentclass[11pt]{article}
\usepackage[utf8]{inputenc}
\usepackage[T1]{fontenc}
\usepackage{amsthm}
\usepackage{enumitem}
\usepackage{amssymb}
\usepackage{amsmath}
\usepackage{amsfonts}
\usepackage[version=4]{mhchem}
\usepackage{stmaryrd}
\usepackage{mathrsfs}
\usepackage{bm}
\usepackage{graphicx}
\usepackage[export]{adjustbox}
\graphicspath{ {./images/} }
\usepackage{algorithm}
\usepackage{algorithmic}
\usepackage{makecell}  % 表格换行


\usepackage{hyperref}
\hypersetup{
    colorlinks=true,     % 启用颜色链接
    linkcolor=blue,     % 内部链接的颜色
    citecolor=blue,      % 引用文献的颜色
    urlcolor=blue,       % URL链接的颜色
    linktoc=red,      % 不影响目录链接颜色
}

\usepackage[a4paper, top=1in, bottom=1in, left=1in, right=1in]{geometry}

\title{
{\bf \huge Notes on M\"untz-Jackson Theorem}
%{\bf \large For M\"untz systems on [0,1]}
}
\author{Huaijin Wang}
\date{December 10, 2024}


\begin{document}


\newtheorem{definition}{Definition}[section]
\newtheorem{property}{Property}[section]
\newtheorem{lemma}{Lemma}[section]
\newtheorem{theorem}{Theorem}[section]
\newtheorem{corollary}{Corollary}[section]
\newtheorem{remark}{Remark}[section]
\newtheorem{example}{Example}[]
\newtheorem{notation}{Notation Declaration}[]
\newtheorem{question}{Question}[]
\newtheorem{exercise}{Exercise}[section]

%\maketitle




\newpage



\setcounter{section}{2}
\setcounter{exercise}{6}
\begin{exercise}
Prove that the consistency and stability leads to the convergence.
\end{exercise}
\begin{proof}
The elliptic problem:
\[
\left\{\begin{array}{l}
L u(x)=f(x), \quad \forall x \in(a, b), \\
u(a)=u(b)=0,
\end{array}\right.
\]
where $L$ is a linear elliptic operator. Let $\{ x_i\}_{i=0}^N$ be a equispaced $(N+1)-$grid in the interval $[a,b]$. That is, $x_i=a+i h$ where $h = (b-a)/N$. Let $L_h$ be a discrete operator to approximate $L$, which leads to the difference schema:
\[
\left\{\begin{array}{l}
L_h u_i=f\left(x_i\right), \quad i=1,2, \cdots, N-1 \\
u_0=u_N=0.
\end{array}\right.
\] 
Let $\mathbf{f} = [f(x_1),\cdots,f(x_{N-1})]^\mathrm{T}$ and $\mathbf{u} = [u(x_1),\cdots,u(x_{N-1})]^{\mathrm{T}}$. We say the difference schema is \textit{stable} if there exists a constant $c$, independend of $h$, such that for all $h$ sufficiently small, it holds 
\[ \|\mathbf{u}\| \leqslant c \| \mathbf{f} \|,\]
where any norm is possible since they are equivalent in finite dimensional space. 
 The truncation error is defined as $R_i = L_h [u(x_i)] - [Lu](x_i)$, where $i=1,\cdots,N-1$. We say the difference schema is \textit{consistent} if
\[
R_i \to 0 \ \text{as} \ h\to 0, \ \text{ for all } \ i=1,\cdots,N-1.
\]
Let $e_i = u(x_i) - u_i$, $i=0,\cdots,N$. Note that $e_0 = e_N=0$, and for $1\leqslant i\leqslant N-1$ we have
\[
R_i =  L_h [u(x_i)] - [Lu](x_i) = L_h [u(x_i)] - f(x_i) =  L_h [u(x_i)] - L_h[u_i] = L_h e_i.
\]
Hence $\{e_i\}_{i=0}^N$ is the solution of the discrete problem:
\[
\left\{\begin{array}{l}
L_h e_i=R_i, \quad i=1,2, \cdots, N-1, \\
e_0=e_N=0 .
\end{array}\right.
\]
By its consistency and stability, we have the convergence, i.e.,
\[
| e_i | \leqslant \|\mathbf{e}\| \leqslant c \|\mathbf{R}\| \to 0 \ \text{as}\ h\to 0, 
\]
where $\mathbf{e} = [e_1,\cdots,e_{N-1}]^{\mathrm{T}}$ and $\mathbf{R} = [R_1,\cdots,R_{N-1}]^{\mathrm{T}}$.
\end{proof}

\begin{exercise}
Consider the variable coefficient equation:
\[
\left\{\begin{array}{l}
L u = f(x), \quad \forall  x\in (a,b), \\
u(a) = u(b) = 0,
\end{array}\right.
\]
where $Lu = -(p u^\prime)^\prime  (x)$, $p_M \geqslant p(x) \geqslant p_0 > 0,\quad \forall x\in [a,b]$. \\
1) Establish an energy inequality. \\
2) Set $p(x) = 1$. Consider the central schema on the non-uniform mesh $\{x_i\}_{i=0}^N$, $h_i = x_i-x_{i-1}$:
\[
\left \{
\begin{aligned}
& L_h u_i = f(x_i),\quad i = 1,2,\cdots,N-1, \\
& u_0 = u_N = 0,
\end{aligned}
\right.
\]
where 
\[
L_h u_i = - \frac{\frac{u_{i+1} - u_i}{h_{i+1}} - \frac{u_i-u_{i-1}}{h_i}}{\frac{h_i+h_{i+1}}{2}}, \quad i = 1,2,\cdots,N-1.
\]
Analyze the truncation error $R_i = L_h[u(x_i)] - [Lu](x_i)$, $i=1,2,\cdots,N-1$ in term of $h = \max_{1\leqslant i \leqslant N} |h_i|$.
\end{exercise}
\begin{proof}[Solution]
~\\
\noindent 1). We know that $(u,L u) = (u, f)$ and leverage integral by parts
\[
(u,Lu) = (u, -(pu^\prime)^\prime) = (p u^\prime, u^\prime) \geqslant p_0 \|u^\prime\|_0^2.
\]
By Cauchy-Schwarz inequality and Poincar\'e inequality, we have
\[
(f,u) \leqslant \|f\|_0 \|u\|_0 \leqslant c \|f\|_0 \|u^\prime\|_0, 
\]
where $c$ is a constant, independent of $f$ and $h$. Hence it is obvious that $\|u^\prime\|_0 \leqslant c/p_0 \|f\|_0$.
~\\
\noindent 2). By Tylor development,
\[
u(x_{i+1}) = u(x_i) + h_{i+1} u^\prime (x_i) + \frac{h_{i+1}^2}{2} u^{\prime \prime}(x_i) + \frac{h^3_{i+1}}{6} u^{\prime \prime\prime}(x_i) + \frac{h^4_{i+1}}{24} u^{(4)}(\xi_i) ,\ \text{for some}\ \xi_i \in (x_i,x_{i+1}).
\]
and 
\[
u(x_{i-1}) = u(x_i) - h_{i} u^\prime (x_i) + \frac{h_{i}^2}{2} u^{\prime \prime}(x_i) - \frac{h^3_{i}}{6} u^{\prime \prime\prime}(x_{i}) + \frac{h^4_{i}}{24} u^{(4)}(\xi_{i-1}),\ \text{for some}\ \xi_{i-1} \in (x_{i-1},x_{i}).
\]
Hence
\[
L_h [u(x_i)] = -u^{\prime\prime}(x_i) - \frac{h_{i+1}^2-h_i^2}{3(h_i+h_{i+1})} u^{\prime \prime \prime }(x_i) - \frac{h_{i+1}^3 u^{(4)}(\xi_i) + h_i^3 u^{(4)} (\xi_{i-1})}{12(h_i+h_{i+1})}.
\]
Hence
\[
\begin{aligned}
|R_i| &= | L_h[u(x_i)] - [Lu](x_i) | = | L_h[u(x_i)] + u^{\prime\prime}(x_i) | \\
& \leqslant \frac{\max_{x\in[a,b]} |u^{\prime\prime\prime}(x)|}{3} |h_{i+1}-h_i|
+ \frac{\max_{x\in[a,b]} |u^{(4)}(x)|}{12} |h_{i+1}^2 - h_{i} h_{i+1} + h_i^2| = O(h).
\end{aligned}
\]
\end{proof}

\setcounter{exercise}{11}
\begin{exercise}
Let $v_h$ be a discrete function defined in $\bar{I}_h$. Prove: \\
1).  Discrete Poincar\'e inequality holds if only $v_0=0$ or $v_N=0$.
~\\
2). If $v_0 = v_N = 0$, then it holds
\[
\|v_h\|^2_0 \leqslant \frac{(b-a)^2}{4} |v_h|_1^2.
\]
\end{exercise}
\begin{proof}
~\\
1). If $v_0 = 0$, we have $v_i = \sum_{j=1}^{i} v_{j,\bar{x}} h_j$ for $i=1,\cdots,N$. Then by Cauchy inequality
\[
v_i^2 = \left(  \sum_{j=1}^{i} v_{j,\bar{x}} h_j \right)^2
\leqslant (x_i-a) \sum_{j=1}^i v_{j,\bar{x}}^2 h_j \leqslant (b-a) \sum_{j=1}^N v_{j,\bar{x}}^2 h_j = (b-a) |v_h|_1^2.
\]
If $v_N=0$, we have $v_i = - \sum_{j=i+1}^{N} v_{j,\bar{x}} h_j$.
~\\
2). It is known that
\[
v_i^2 \leq \frac{\left(x_i-a\right)\left(b-x_i\right)}{b-a}\left|v_h\right|_1^2 \leq \frac{b-a}{4}\left|v_h\right|_1^2, \quad \forall i \in \bar{I}_h .
\]
Therefore
\[
\left\|v_h\right\|_0^2=\sum_{\bar{I}_h} v_i^2 h_i \leq \frac{b-a}{4}\left|v_h\right|_1^2 \sum_{\bar{I}_h} \bar{h}_i \leq \frac{(b-a)^2}{4}\left|v_h\right|_1^2 .
\]
\end{proof}



\section*{Appendix: Notations for Discrete Representation}
 Let $I = [a,b]$. We define the discrete grid points as
\[
a=x_0<x_1<\cdots<x_N = b.
\]
We introduce the following sets:
\[
I_h = \{x_1,\cdots,x_{N-1}\}, \ \bar{I}_h = \{x_0,x_1,\cdots, x_N\}, \ I_h^+ = \{x_1,\cdots,x_N\}.
\]
The grid spacing is defined as
\[
h_i = x_{i}- x_{i-1}, \quad i=1,\cdots,N.
\]
Additionally, we define the averaged grid spacing:
\[
\begin{aligned}
& \bar{h}_i = \frac{1}{2} (h_i+h_{i+1}), \ i=1,\cdots,N-1,\\
& \bar{h}_0  = \frac{1}{2} h_1, \quad \bar{h}_N = \frac{1}{2} h_N.
\end{aligned}
\]
A discrete function defined on $\bar{I}_h$ is denoted as 
\[
v_h = \{v_0,v_1,\cdots, v_N \}.
\]
We define the following difference operators:
\[
\begin{aligned}
& (v_i)_{\bar{x}} := v_{i,\bar{x}} : = \frac{v_i-v_{i-1}}{h_i}, \ i =1,\cdots,N, \\
& (v_i)_x := v_{i, x} := \frac{v_{i+1} - v_i}{h_{i+1}},\ i=0,\cdots,N-1, \\
&  (v_i)_{\hat{x}} := v_{i, \hat{x}} := \frac{v_{i+1} - v_i}{\bar{h}_{i}},\ i=0,\cdots,N-1.
\end{aligned}
\]
The discrete inner products are given by
\[
(u_h, v_h)_{I_h} = \sum_{i=1}^{N-1} u_i v_i \bar{h}_i, \
(u_h, v_h)_{\bar{I}_h} = \sum_{i=0}^{N} u_i v_i \bar{h}_i, \
(u_h, v_h)_{I^+_h} = \sum_{i=1}^{N} u_i v_i h_i.
\]
We define the discrete norms as follows:
\[
\begin{aligned}
& \|v_h\|_c := \max_{\bar{I}_h} |v_i|,\ \|v_h\|_0 := (v_h,v_h)_{\bar{I}_h}^{1/2}, \\
& |v_h|_1 := ((v_h)_{\bar{x}}, (v_h)_{\bar{x}})_{I_h^+}^{1/2}, \ \|v_h\|_1^2 = \|v_h\|_0^2 + |v_h|_1^2.
\end{aligned}
\]
The discrete integral by parts:
\[
\sum_{i=m+1}^n v_i (w_i)_{\bar{x}} h_i = - \sum_{i=m}^{n-1} (v_i)_x w_i h_{i+1} + v_n w_n - v_m w_m,\ \text{for some} \ 0\leqslant m < n \leqslant N.
\]
The discrete Green formula:
\[
\sum_{i=m+1}^{n-1} \left( (u_i)_{\bar{x}} \right )_{\hat{x}} v_i \bar{h}_i = - \sum_{i=m+1}^n (u_i)_{\bar{x}} (v_i)_{\bar{x}} h_i + (u_n)_{\bar{x}} v_n - (u_m)_x v_m,\ \text{for some} \ 0\leqslant m < n \leqslant N.
\]
The discrete Cauchy-Schwarz inequality states that
\[
|(u_h, v_h)_{\bar{I}_h}| \leqslant (u_h, u_h)_{\bar{I}_h}^{1/2} (v_h, v_h)_{\bar{I}_h}^{1/2}.
\]
If $v_0 = 0$ (or $v_N=0$ or $v_0=v_N=0$), the discrete Poincar\'e inequality holds:
\[
\|v_h\|_c \leqslant C |v_h|_1, \quad \|v_h\|_0 \leqslant C |v_h|_1,
\]
where $C$ is a constant depending only on $a$ and $b$.


\end{document}

