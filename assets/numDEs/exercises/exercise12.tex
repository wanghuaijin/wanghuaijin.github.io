
% This LaTeX was auto-generated from MATLAB code.
% To make changes, update the MATLAB code and republish this document.

\documentclass[11pt]{article}
\usepackage[utf8]{inputenc}
\usepackage[T1]{fontenc}
\usepackage{amsthm}
\usepackage{enumitem}
\usepackage{amssymb}
\usepackage{amsmath}
\usepackage{amsfonts}
\usepackage[version=4]{mhchem}
\usepackage{stmaryrd}
\usepackage{mathrsfs}
\usepackage{bm}
\usepackage{graphicx}
\usepackage[export]{adjustbox}
\graphicspath{ {./images/} }
\usepackage{algorithm}
\usepackage{algorithmic}
\usepackage{makecell}  % 表格换行


\usepackage{hyperref}
\hypersetup{
    colorlinks=true,     % 启用颜色链接
    linkcolor=blue,     % 内部链接的颜色
    citecolor=blue,      % 引用文献的颜色
    urlcolor=blue,       % URL链接的颜色
    linktoc=red,      % 不影响目录链接颜色
}

\usepackage[a4paper, top=1in, bottom=1in, left=1in, right=1in]{geometry}

\title{
{\bf \huge Notes on M\"untz-Jackson Theorem}
%{\bf \large For M\"untz systems on [0,1]}
}
\author{Huaijin Wang}
\date{December 10, 2024}


\begin{document}


\newtheorem{definition}{Definition}[section]
\newtheorem{property}{Property}[section]
\newtheorem{lemma}{Lemma}[section]
\newtheorem{theorem}{Theorem}[section]
\newtheorem{corollary}{Corollary}[section]
\newtheorem{remark}{Remark}[section]
\newtheorem{example}{Example}[]
\newtheorem{notation}{Notation Declaration}[]
\newtheorem{question}{Question}[]
\newtheorem{exercise}{Exercise}[section]
\newtheorem{exercise*}{Exercise}[]

%\maketitle




\newpage



\begin{exercise*}
Let $\Omega = \{ \mathbf{x}\in \mathbb{R}^2: |\mathbf{x}|<1/2\}$. Define 
\[
v(\mathbf{x}) = \left( \log |\mathbf{x}| \right )^k, \ \forall \mathbf{x}\in \Omega \backslash \{0\},\ 0<k<1/2.
\] 
Prove $v\in H^1(\Omega)$.
\end{exercise*}
\begin{proof}
It is clear that $v\in L^2(\Omega)$ and 
\[
\nabla v (\mathbf{x}) = k (-\log |\mathbf{x}| )^{k-1} \frac{\mathbf{x}}{|\mathbf{x}|^2},
\]
which is also in $L^2(\Omega)$, i.e., $\| \nabla v \|_0 < \infty$.
\end{proof}


\begin{exercise*}
1). If $K$ is a rectangle, $P(K) \hat{=} Q_1(K) = \mathrm{span}\{1,x,y,xy\}$, $\Sigma_K = \{\text{mid-points of four sides}\}$. Prove that $(K,p(K),\Sigma_K)$ is not a finite element. \\
2). If $K$ is a rectangle, $P(K) \hat{=} Q_1^T(K) = \mathrm{span}\{1,x,y,x^2-y^2\}$, $\Sigma_K = \{\text{mid-points of four sides}\}$. Prove that $(K,P(K), \Sigma_K)$ is a finite element.
\end{exercise*}

\end{document}

