
% This LaTeX was auto-generated from MATLAB code.
% To make changes, update the MATLAB code and republish this document.

\documentclass[11pt]{article}
\usepackage[utf8]{inputenc}
\usepackage[T1]{fontenc}
\usepackage{amsthm}
\usepackage{enumitem}
\usepackage{amssymb}
\usepackage{amsmath}
\usepackage{amsfonts}
\usepackage[version=4]{mhchem}
\usepackage{stmaryrd}
\usepackage{mathrsfs}
\usepackage{bm}
\usepackage{graphicx}
\usepackage[export]{adjustbox}
\graphicspath{ {./images/} }
\usepackage{algorithm}
\usepackage{algorithmic}
\usepackage{makecell}  % 表格换行


\usepackage{hyperref}
\hypersetup{
    colorlinks=true,     % 启用颜色链接
    linkcolor=blue,     % 内部链接的颜色
    citecolor=blue,      % 引用文献的颜色
    urlcolor=blue,       % URL链接的颜色
    linktoc=red,      % 不影响目录链接颜色
}

\usepackage[a4paper, top=1in, bottom=1in, left=1in, right=1in]{geometry}

\title{
{\bf \huge Notes on M\"untz-Jackson Theorem}
%{\bf \large For M\"untz systems on [0,1]}
}
\author{Huaijin Wang}
\date{December 10, 2024}


\begin{document}


\newtheorem{definition}{Definition}[section]
\newtheorem{property}{Property}[section]
\newtheorem{lemma}{Lemma}[section]
\newtheorem{theorem}{Theorem}[section]
\newtheorem{corollary}{Corollary}[section]
\newtheorem{remark}{Remark}[section]
\newtheorem{example}{Example}[]
\newtheorem{notation}{Notation Declaration}[]
\newtheorem{question}{Question}[]
\newtheorem{exercise}{Exercise}[section]

%\maketitle




\newpage

\setcounter{section}{2}

\setcounter{exercise}{2}
\begin{exercise}
Determine the absolute stability region of the Crank-Nicolson schema and Leapfrog schema.
\end{exercise}
\begin{proof}[Solution]
Model problem:
\[
\frac{\mathrm{d} u}{\mathrm{d} t} = \lambda u,\quad \text{where } \lambda \ \text{is a constant.}
\]
~\\
\noindent $\bullet$ Crank-Nicolson schema:
\[
u^{n+1} = u^n + \frac{\lambda h( u^{n+1} + u^n)}{2} \Rightarrow u^{n+1} = \frac{2+h\lambda}{2-h\lambda} u^n.
\]
Let $z=\lambda h$. The absolute stability region of this schema is
\[
\left \{z: \Big | \frac{2+z}{2-z} \Big | \leqslant 1  \right \} = \left \{z:  \mathrm{Re}(z) \leqslant 0 \right \} .
\]
In fact, if we suppose that $z=a+b \mathrm{i}$, then $|2+z| \leqslant |2-z|$ is equivalent to
\[
(a+2)^2 + b^2 \leqslant (a-2)^2+b^2 \Leftrightarrow a\leqslant 0.
\]
~\\
\noindent $\bullet$ Leapfrog schema:
\[
u^{n+1} = 2h\lambda u^n + u^{n-1}.
\]
Let $z=\lambda h$, then its characteristic polynomial is $\rho(\xi) = \xi^2-2z\xi-1$. Hence roots of $\rho(\xi)$ are $\xi_{1,2} = z\pm \sqrt{z^2+1}$. The schema is absolute stable if and only if the roots satisfy the condition (see \cite[Definition 7.1, p. 153]{leveque2007}):
\[
\begin{aligned}
& |\xi_j | \leqslant 1,\quad j=1,2.\\
& \text{If } \xi_1 = \xi_2, \text{ then } |\xi_1|<1.
\end{aligned}
\]
We note that $|\xi_1| |\xi_2| = 1$, then the only possibility to satisfy the root condition is $|\xi_1| = |\xi_2|=1$, which leads to that $z$ is the pure imaginary number. By setting $z = b\mathrm{i}$, we obtain $b\in (-1,1)$. Hence the absolute stability region is $\{z = b\mathrm{i}: b\in (-1,1)\}$.
\end{proof}



\begin{thebibliography}{99}
\bibitem[LeVeque (2007)]{leveque2007} \href{https://wanghuaijin.github.io/assets/numDEs/Leveque2007.pdf}{LeVeque R J. Finite difference methods for ordinary and partial differential equations: steady-state and time-dependent problems[M]. Society for Industrial and Applied Mathematics, 2007.}
\end{thebibliography}


\end{document}

