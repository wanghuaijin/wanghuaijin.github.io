
% This LaTeX was auto-generated from MATLAB code.
% To make changes, update the MATLAB code and republish this document.

\documentclass[11pt]{article}
\usepackage[utf8]{inputenc}
\usepackage[T1]{fontenc}
\usepackage{amsthm}
\usepackage{enumitem}
\usepackage{amssymb}
\usepackage{amsmath}
\usepackage{amsfonts}
\usepackage[version=4]{mhchem}
\usepackage{stmaryrd}
\usepackage{mathrsfs}
\usepackage{bm}
\usepackage{graphicx}
\usepackage[export]{adjustbox}
\graphicspath{ {./images/} }
\usepackage{algorithm}
\usepackage{algorithmic}
\usepackage{makecell}  % 表格换行


\usepackage{hyperref}
\hypersetup{
    colorlinks=true,     % 启用颜色链接
    linkcolor=blue,     % 内部链接的颜色
    citecolor=blue,      % 引用文献的颜色
    urlcolor=blue,       % URL链接的颜色
    linktoc=red,      % 不影响目录链接颜色
}

\usepackage[a4paper, top=1in, bottom=1in, left=1in, right=1in]{geometry}

\title{
{\bf \huge Notes on M\"untz-Jackson Theorem}
%{\bf \large For M\"untz systems on [0,1]}
}
\author{Huaijin Wang}
\date{December 10, 2024}


\begin{document}


\newtheorem{definition}{Definition}[section]
\newtheorem{property}{Property}[section]
\newtheorem{lemma}{Lemma}[section]
\newtheorem{theorem}{Theorem}[section]
\newtheorem{corollary}{Corollary}[section]
\newtheorem{remark}{Remark}[section]
\newtheorem{example}{Example}[]
\newtheorem{notation}{Notation Declaration}[]
\newtheorem{question}{Question}[]
\newtheorem{exercise}{Exercise}[]

%\maketitle




\newpage

\setcounter{exercise}{0}
\begin{exercise}[See Remark 1.3]
Prove 
\[
\begin{aligned}
& \|u-u_I\|_0 \leqslant C h^2\|u^{\prime\prime}\|_0,\\
& \|u^\prime - u_I^\prime\|_0 \leqslant C h \|u^{\prime\prime}\|_0.
\end{aligned}
\]
\end{exercise}
\begin{proof}
$\bullet$ \textbf{Step 1}. Show that for $I=(0,1)$ and $f\in H^2(I)\cap H_0^1(I)$, we have 
\[
\int_0^1 f(X)^2 \mathrm{d} X \leqslant \int_0^1 f^\prime(X)^2 \mathrm{d} X,\ \text{and}\
\int_0^1 f^\prime(X)^2 \mathrm{d} X \leqslant \int_0^1 f^{\prime\prime} (X)^2 \mathrm{d} X.
\]
Since $f(0)=f(1)=0$, there exists $X_0\in I$ such that $f^\prime(X_0)=0$. Thus
\[
f(X) = \int_0^X f^\prime (X) \mathrm{d} X,\ \text{and}\ 
f^\prime(X) = \int_{X_0}^X f^{\prime \prime} (X) \mathrm{d} X.
\]
Therefore, the conclusion is clear by Cauchy-Schwarz inequality.

$\bullet$ \textbf{Step 2.} Make variable change to subinterval $I_{j+1}=(x_{j},x_{j+1})$ by $x = x_{j-1} + X(x_j-x_{j-1})$, we have
\[
\int_{x_{j-1}}^{x_{j}} \tilde{f}(x)^2 \mathrm{d} x \leqslant (x_j-x_{j-1})^2 \int_{x_{j-1}}^{x_j} \tilde{f}^\prime (x)^2 \mathrm{d} x,\ \text{and}\
\int_{x_{j-1}}^{x_j} \tilde{f}^{\prime} (x)^2 \mathrm{d} x \leqslant 
(x_j-x_{j-1})^2 \int_{x_{j-1}}^{x_j} \tilde{f}^{\prime\prime}(x)^2 \mathrm{d} x,
\]
where $\tilde{f}(x) = f(X) = f( \frac{x-x_{j-1}}{x_j-x_{j-1}})\in H^2(I_j) \cap H_0^1(I_j)$. The conclusion is obvious since 
\[
\mathrm{d} x = (x_j-x_{j-1}) \mathrm{d} X,\
f^\prime(X) = \tilde{f}^\prime(x) (x_j-x_{j-1}),\
f^{\prime\prime}(X) = \tilde{f}^{\prime\prime} (x) (x_j-x_{j-1})^2.
\]

$\bullet$ \textbf{Step 3.} Let $\tilde{f}(x) = u(x) - u_I(x)$. It is obvious that $\tilde{f}(x_i) = 0$ for $i=0,\cdots,N+1$. Thus $\left( \tilde{f}\big |_{I_j} \right)^{\prime\prime} = u^{\prime\prime}$ and $\tilde{f} \big |_{I_{j}} \in H^2(I_j)\cap H_0^1 (I_j)$ for $j=1,\cdots,N+1$, and we have
\[
\int_{I_j} (u-u_I)^2 \mathrm{d} x \leqslant h_j^2 \int_{I_j} (u^\prime-u^\prime_I)^2 \mathrm{d} x,\ 
\int_{I_j} (u^\prime - u^\prime_I)^2 \mathrm{d} x \leqslant h_j^2 \int_{I_j} (u^{\prime\prime})^2 \mathrm{d} x,
\]
both of which leads to
\[
\|u-u_I\|_0 \leqslant h \|u^\prime - u_I^\prime\|_0,\ \text{and}\
\|u^\prime - u_I^\prime\|_0 \leqslant h \|u^{\prime\prime}\|_0.
\]

\end{proof}

\begin{exercise}
Consider the mixed boundary problem
\[
\left\{
\begin{aligned}
&-u^{\prime \prime}=f, \quad x \in I:=(0,1), \\
&u(0)=0,\ u^{\prime}(1)=\beta ,
\end{aligned} \right.
\]
where $\beta\in\mathbb{R}$ and $f\in L^2(I)$. Construct and analyze $P_1-$FEM for this problem.
\end{exercise}
\begin{proof}

$\bullet$ Variational form. Let $V = \{v\in H^1(I): v(0)=0\}$, the bilinear form  $a(u,v) = (u^\prime, v^\prime)$, and the functional $\mathcal{F}(v) = (f,v) + \beta v(1)$. Then the variational problem reads
\[
\left\{
\begin{aligned}
&\text{Find}\ u\in V \ \text{such that} \\
& a(u,v) = \mathcal{F}(v), \quad \forall v\in V,
\end{aligned}
\right.
\]
which is clearly equivalent to the strong problem.

$\bullet$ Galerkin Approximation. Let $V_h$ be a subspace of $V$ with finite dimension. Then the Galerkin approximation reads
\[
\left\{
\begin{aligned}
&\text{Find}\ u_h \in V_h \ \text{such that} \\
& a(u_h, v_h) = \mathcal{F}(v_h), \quad \forall v_h \in V_h.
\end{aligned}
\right.
\]

$\bullet$ $P1-$FEM. By construct the space of piecewise linear polynomials $X_h^1$ and its basis $\varphi_0,\cdots,\varphi_{N+1}$, shown in the Appendix, we let the finite element space $V_h=X_h^1 \cap V$, then 
\[
V_h = \mathrm{span}\{\varphi_1,\cdots,\varphi_{N+1}\}.
\]

$\bullet$ FEM Implementation.
Let $u_h = \sum_{j=1}^{N+1} u_j \varphi_j(x)$, then 
\[
\sum_{j=1}^{N+1} u_j a(\varphi_j, \varphi_i) = \mathcal{F}(\varphi_i),\quad i=1,\cdots,N+1.
\]
Let $\mathbf{A} = (a_{i, j})$ be the $(N+1)\times(N+1)$ matrix with its entries $a_{i,j} = a(\varphi_j,\varphi_i)$. Then we have
\[
\begin{aligned}
&a_{N+1,N+1} = \frac{1}{h_{N+1}},\ a_{j,j} = \frac{1}{h_j} + \frac{1}{h_{j+1}}, \quad j=1,\cdots,N, \\
& a_{j,j-1} = -\frac{1}{h_{j}}, \ j=1,\cdots,N+1, \\
& a_{i,j} = 0,\ \text{if}\ |i-j|\geqslant 2.
\end{aligned}
\]
Thus
\[
\frac{1}{h}
\begin{bmatrix}
2 & -1 & & & \\
-1 & 2 & -1 &&  \\
 & \ddots &\ddots & \ddots & \\
 & & -1 & 2 & -1 \\
 & & & -1 & 1 \\
\end{bmatrix}
\begin{bmatrix}
u_1 \\
u_2 \\
\vdots \\
u_N \\
u_{N+1}
\end{bmatrix}
= 
\begin{bmatrix}
(f,\varphi_1) \\
(f,\varphi_2) \\
\vdots \\
(f,\varphi_N) \\
(f,\varphi_{N+1}) + \beta \\
\end{bmatrix}.
\] 

$\bullet$ Error Estimate. We denote $u_I$ being the interpolation of $u$ into $V_h$, then it is clear that 
\[
\|u-u_I\|_0 \leqslant Ch \|u^\prime - u_I^\prime\|_0 \leqslant Ch^2 \|u^{\prime\prime}\|_0.
\]
 We know $a(u-u_h, v_h) = 0$ for any $v_h\in V_h$. Then
 \[
 \|u^\prime-u_h^\prime\|_0^2 = a(u-u_h, u-u_h) = a(u-u_h, u-v_h) \leqslant \|u^\prime-u_h^\prime\|_0 \|u^\prime-v_h^\prime\|_0,\quad \forall v_h\in V_h,
 \]
 which leads to
\[
\|u^\prime-u^\prime_h\|_0 \leqslant \inf_{v_h\in V_h} \|u^\prime - v^\prime_h\|_0 \leqslant \|u^\prime - u^\prime_I \|_0 \leqslant Ch \|u^{\prime\prime}\|_0.
\]
In the following, we derive the estimate for $\|u-u_h\|_0$.

 Dual problem: given $r\in L^2(I)$,
 \[
 \left\{
 \begin{aligned}
 &\text{Find}\ \varphi(r)\in V\ \text{such that} \\
 & a(v,\varphi(r)) = (r,v),\quad \forall v\in V.
 \end{aligned}
 \right.
 \]
The dual problem admits a unique solution $\varphi(r)$ since $a(\cdot,\cdot)$ is continuous and coercive.  Moreover, we have
 \[
 a(v,\varphi(r)) = (r,v),\quad \forall v\in C_0^\infty(I),
 \]
if we suppose $\varphi(r)\in H^2(I)$, which gives $(-\varphi^{\prime\prime} (r), v) = (r,v),\ \forall v\in C_0^\infty (I)$.
Since $C_0^\infty(I)$ is dense in $L^2(I)$, we have
\[
\|\varphi^{\prime\prime}(r)\|_0 = \sup_{v\in L^2(I), \ v\neq 0} \frac{(\varphi, v)}{\|v\|_0}
= \sup_{v\in L^2(I), \ v\neq 0} \frac{(r,v)}{\|v\|_0} = \|r\|_0.
\]
Thus we denote $\varphi_I(r)$ being the interpolation of $\varphi(r)$ into $V_h$ and obtain 
\[
\begin{aligned}
\|u-u_h\|_0 & = \sup_{r\in L^2(I), \ r\neq 0} \frac{(r,u-u_h)}{\|r\|_0} = 
\sup_{r\in L^2(I), \ r\neq 0}  \frac{a(u-u_h, \varphi(r))}{\|r\|_0} \\
& = \sup_{r\in L^2(I), \ r\neq 0}  \frac{a(u-u_h, \varphi(r) - \varphi_I (r))}{\|r\|_0} \\
 & \leqslant \sup_{r\in L^2(I), \ r\neq 0}  \frac{\|u^\prime-u^\prime_h\|_0 \|\varphi^\prime(r) - \varphi^\prime_I(r)\|_0 }{\|r\|_0} \\
 & \leqslant C h\|u^\prime-u^\prime_h\|_0 \sup_{r\in L^2(I), \ r\neq 0} \frac{\|\varphi^{\prime\prime}(r)\|_0}{\|r\|_0}  \\
 & \leqslant C h\|u^\prime-u^\prime_h\|_0.
\end{aligned}
\]
\end{proof}











\section*{Appendix}
Let $I=(0,1)$ and $\{x_n\}_{n=0}^{N+1}$ be a grid on $I$ such that $0=x_0<x_1<\cdots<x_N<x_{N+1}=1$. Denote by each subintervals (or elements) $I_n=(x_{n-1},x_n)$ for $1\leqslant n\leqslant N+1$ of length $h_n=x_n-x_{n-1}$. Let $h = \max_{1\leqslant n \leqslant N+1} h_n$.

The piecewise linear polynomials on such grid is denoted by
\[
X_h^1 := \{v\in C(\bar{I}): v\big |_{I_{j+1}} \in \mathbb{P}_1, \ j=0,\cdots,N \}.
\]
We construct a nodal basis for $X_h^1$, which is based on nodes in every element (how many nodes in every element depends on the degree of freedom, or the degree of polynomials required parameters to be determined). 
\[
\varphi_0 (x) = \left \{
\begin{aligned}
&\frac{x_1-x}{x_1-x_0}, &\quad x\in I_1,\\
& 0, &\quad \text{else},
\end{aligned}
\right.
\qquad 
\varphi_{N+1}(x) = \left\{
\begin{aligned}
& \frac{x-x_N}{x_{N+1}-x_N}, & \quad x\in I_{N+1}, \\
& 0,& \text{else} ,
\end{aligned}
\right .
\]
\[
\varphi_n(x) = \left\{
\begin{aligned}
& \frac{x-x_{n-1}}{x_n-x_{n-1}}, & \quad x\in I_n, \\
& \frac{x_{n+1} - x}{x_{n+1} - x_n}, & \quad x\in I_{n+1}, \\
& 0,&\quad \text{else}.
\end{aligned}
\right.
\]
Clearly, we have $X_h^1 = \mathrm{span}\{\varphi_0,\varphi_1,\cdots,\varphi_{N+1}\}$.

For any $u\in C(\bar{I})$, its interpolation into $X_h^1$ is denoted by $u_I(x)$. Clearly, we have $u_I(x) = \sum_{i=0}^{N+1} u(x_i) \varphi_i(x)$ and 
\[
u_I \big |_{I_{j+1}} = u(x_j) \varphi_j(x) + u(x_{j+1}) \varphi_{j+1}(x) =
u(x_j) \frac{x_{j+1}-x}{x_{j+1} - x_j} + u(x_{j+1}) \frac{x-x_j}{x_{j+1}-x_j}.
\]

\end{document}

