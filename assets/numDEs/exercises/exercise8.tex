
% This LaTeX was auto-generated from MATLAB code.
% To make changes, update the MATLAB code and republish this document.

\documentclass[11pt]{article}
\usepackage[utf8]{inputenc}
\usepackage[T1]{fontenc}
\usepackage{amsthm}
\usepackage{enumitem}
\usepackage{amssymb}
\usepackage{amsmath}
\usepackage{amsfonts}
\usepackage[version=4]{mhchem}
\usepackage{stmaryrd}
\usepackage{mathrsfs}
\usepackage{bm}
\usepackage{graphicx}
\usepackage[export]{adjustbox}
\graphicspath{ {./images/} }
\usepackage{algorithm}
\usepackage{algorithmic}
\usepackage{makecell}  % 表格换行


\usepackage{hyperref}
\hypersetup{
    colorlinks=true,     % 启用颜色链接
    linkcolor=blue,     % 内部链接的颜色
    citecolor=blue,      % 引用文献的颜色
    urlcolor=blue,       % URL链接的颜色
    linktoc=red,      % 不影响目录链接颜色
}

\usepackage[a4paper, top=1in, bottom=1in, left=1in, right=1in]{geometry}

\title{
{\bf \huge Notes on M\"untz-Jackson Theorem}
%{\bf \large For M\"untz systems on [0,1]}
}
\author{Huaijin Wang}
\date{December 10, 2024}


\begin{document}


\newtheorem{definition}{Definition}[section]
\newtheorem{property}{Property}[section]
\newtheorem{lemma}{Lemma}[section]
\newtheorem{theorem}{Theorem}[section]
\newtheorem{corollary}{Corollary}[section]
\newtheorem{remark}{Remark}[section]
\newtheorem{example}{Example}[]
\newtheorem{notation}{Notation Declaration}[]
\newtheorem{question}{Question}[]
\newtheorem{exercise}{Exercise}[section]

%\maketitle




\newpage

\setcounter{section}{1}
\setcounter{exercise}{1}
\begin{exercise}
Prove some alternative forms of the Poincar\'e inequality:
\[
\begin{aligned}
& \|v \|_{L^\infty} \leqslant c_1 \| v^\prime\|_0, \quad \forall v\in \{ v\in H^1(I), \  v(0) = 0 \}. \\
& \|v \|_0 \leqslant c_2 \| v^\prime \|_0,\quad \forall v\in \{ v\in H^1(I),\ v(0)=0\}. 
\end{aligned}
\]
\end{exercise}
\begin{proof}
Let $V = \{ v\in H^1(I), \  v(0) = 0 \}$ and $U = \{v\in C^\infty(I),\ v(0)=0\}$. Then $U$ is dense in $V$ with respect to $\|\cdot\|_1$, i.e., $\forall v\in V$, there exists $\{v_n\} \subset U$ such that 
\[
\lim_{n\to\infty} \|v_n-v\|_1 = 0.
\]
Thus
\[
\begin{aligned}
& \|v - v_n\|_{L^\infty} \leqslant \|v-v_n\|_1 \to 0, \ \text{as}\ n\to \infty, \\
& \|v - v_n\|_0 \leqslant \|v-v_n\|_1 \to 0, \ \text{as}\ n\to \infty, \\
& \|v^\prime- v^\prime_n\|_{0} = |v-v_n|_1 \leqslant \|v-v_n\|_1 \to 0, \ \text{as}\ n\to \infty. \\
\end{aligned}
\]
Therefore, it is sufficient to show that the inequalities hold for any $v\in U$, which is obvious since
\[
|v(x)| = \left| \int_0^x v^\prime(x) \mathrm{d} x \right| \leqslant 
\left(\int_0^x 1^2 \mathrm{~d} t\right)^{\frac{1}{2}}\left(\int_0^x\left|v_n^{\prime}(t)\right|^2 \mathrm{~d} t\right)^{\frac{1}{2}} \leqslant  \|v^\prime\|_0.
\]
\end{proof}

\begin{exercise}
Consider the boundary value problem:
\begin{equation}
\left\{\begin{array}{l}
-u^{\prime \prime}(x)=f(x), \quad x \in I:=(0,1), \\
u(0)=u^{\prime}(1)=0,
\end{array}\right.
\label{eq:2}
\tag{2}
\end{equation}
where $f$ is a given continuous function. Let
\[
V = \{v: v \ \text{and} \ v^\prime\ \text{are square integrable on} \ [0,1],\ \text{and} \ v(0)=0\}.
\]
The corresponding minimization problem of \eqref{eq:2} reads: Find $u\in V$, such that
\begin{equation}
\mathcal{F}(u)=\inf _{v \in V} \mathcal{F}(v),
\label{eq:3}
\tag{3}
\end{equation}
where $\mathcal{F}$ is defined as:
\[
\mathcal{F}(v)=\frac{1}{2}\left(v^{\prime}, v^{\prime}\right)-(f, v), \quad \forall v \in V.
\]
The corresponding variational problem of \eqref{eq:2}: Find $u\in V$, such that
\begin{equation}
\left(u^{\prime}, v^{\prime}\right)=(f, v), \quad \forall v \in V .
\label{eq:4}
\tag{4}
\end{equation}
Prove that:
~\\
1) All three problems \eqref{eq:2}, \eqref{eq:3} and \eqref{eq:4} are equivalent. \\
2) The problem \eqref{eq:2} admits one unique solution. \\
3) The solution of \eqref{eq:4} is unique.
\end{exercise}
\begin{proof}
~\\
1). 
~\\
$\bullet$ \eqref{eq:2} $\Leftrightarrow$ \eqref{eq:4}. Suppose that $u$ is the solution of \eqref{eq:2}. Then $\forall v\in V$, we have $(-u^{\prime\prime},v) = (f,v)$ and
\[
 (-u^{\prime\prime}, v) = (u^\prime, v^\prime) - u^\prime(1) v(1) + u^\prime(0) v(0) = (u^\prime, v^\prime),
\]
which leads to that $u$ is the solution of \eqref{eq:4}. Conversely, we suppose that $u$ is the solution of \eqref{eq:4}. Then we have
\[
(u^\prime, v^\prime) = (f,v),\quad \forall v\in C_0^\infty (I),
\]
which leads to
\[
(-u^{\prime \prime}, v) = (f,v), \quad \forall v\in C_0^\infty(I),
\]
where $u^{\prime\prime}$ is the derivative of $u^\prime$ in the distribution sense. Thus $- u^{\prime \prime} = f$ in the distribution sense. For the boundary conditions, $u(0)=0$ is obvious since $u\in V$, and $u^\prime(1) = 0$ follows from integral by parts, i.e., 
\[
(u^\prime, v^\prime) = (-u^{\prime\prime},v)+ u^\prime(1) v(1),\quad \forall v\in V.
\]
$\bullet$ \eqref{eq:3} $\Leftrightarrow$ \eqref{eq:4}. Suppose that $u$ is the solution of \eqref{eq:3}. Then $\forall \alpha \in \mathbb{R}$ and $v\in V$, we have $\mathcal{F}(u) \leqslant \mathcal{F}( u+\alpha v)$, which is convex over $\alpha$ and attains its minimum at $\alpha=0$. Thus
\[
\frac{\mathrm{d} \mathcal{F}(u+\alpha v)}{\mathrm{d} \alpha} \Big |_{\alpha=0} = 0,
\]
which leads to $(u^\prime, v^\prime) = (f,v)$. Conversely, we suppose that $u$ is the solution of \eqref{eq:4}. For any $v\in V$, we set $w = v-u$. Then
\[
\begin{aligned}
\mathcal{F}(v) & = \mathcal{F}(u+w) = \frac{1}{2} (u^\prime+w^\prime, u^\prime+w^\prime) - (f, u+w) \\
& = \frac{1}{2} (u^\prime, u^\prime) - (f,u) + (w^\prime, w^\prime) + (u^\prime, w^\prime) - (f,w) \\
& = \mathcal{F}(u) + \|w^\prime\|_0^2 \geqslant \mathcal{F}(u).
\end{aligned}
\]
This implies that $\mathcal{F} (u) = \inf_{v \in V}{\mathcal{F}(v)}$.
~\\ 
2). If we set
\[
u(x) = -\int_0^x  \int_0^t f(s) \mathrm{d} s \mathrm{d} t + x \int_0^1 f(s) \mathrm{d} s,
\]
then $u$ is the solution of \eqref{eq:2}. Suppose that $u_1$ and $u_2$ are two solutions of \eqref{eq:2}. Let $\tilde{u} = u_1-u_2$, then
\[
\left\{
\begin{aligned}
&-\tilde{u}^{\prime \prime} = 0, \ x\in I, \\
&\tilde{u}(0)=\tilde{u}^\prime(1) = 0.
\end{aligned}
\right.
\]
Thus $\tilde{u}$ is a bounded harmonic function. By Liouville's theorem, which states any bounded harmonic function is constant, we have $\tilde{u}=0$.
~\\
3). Let $a(u,v) = (u^\prime, v^\prime)$. By Lax-Milgram Lemma, it is sufficient to check that
\begin{enumerate}[label=(\roman*)]
\item $f\in L^2(I)$.
\item $a(\cdot,\cdot)$ is a bilinear form.
\item $a(\cdot,\cdot)$ is continuous, i.e.,
\[
|a(u,v)| = |(u^\prime,v^\prime)|\leqslant \|u\|_1 \|v\|_1,\quad \forall u,v\in V.
\]
\item $a(\cdot,\cdot)$ is coercive. By Poincar\'e inequality:
\[
\|v\|_1^2 = \|v\|_0^2 + \|v^\prime\|_0^2 \leqslant (1+c^2_p) \|v^\prime\|_0^2,\quad \forall v\in V,
\]
which leads to
\[
a(v,v) = \| v^\prime \|_0^2 \geqslant \frac{1}{1+c_p^2} \|v\|_1^2.
\]
\end{enumerate}
\end{proof}


\end{document}

