
% This LaTeX was auto-generated from MATLAB code.
% To make changes, update the MATLAB code and republish this document.

\documentclass[11pt]{article}
\usepackage[utf8]{inputenc}
\usepackage[T1]{fontenc}
\usepackage{amsthm}
\usepackage{enumitem}
\usepackage{amssymb}
\usepackage{amsmath}
\usepackage{amsfonts}
\usepackage[version=4]{mhchem}
\usepackage{stmaryrd}
\usepackage{mathrsfs}
\usepackage{bm}
\usepackage{graphicx}
\usepackage[export]{adjustbox}
\graphicspath{ {./images/} }
\usepackage{algorithm}
\usepackage{algorithmic}
\usepackage{makecell}  % 表格换行


\usepackage{hyperref}
\hypersetup{
    colorlinks=true,     % 启用颜色链接
    linkcolor=blue,     % 内部链接的颜色
    citecolor=blue,      % 引用文献的颜色
    urlcolor=blue,       % URL链接的颜色
    linktoc=red,      % 不影响目录链接颜色
}

\usepackage[a4paper, top=1in, bottom=1in, left=1in, right=1in]{geometry}

\title{
{\bf \huge Notes on M\"untz-Jackson Theorem}
%{\bf \large For M\"untz systems on [0,1]}
}
\author{Huaijin Wang}
\date{December 10, 2024}


\begin{document}


\newtheorem{definition}{Definition}[section]
\newtheorem{property}{Property}[section]
\newtheorem{lemma}{Lemma}[section]
\newtheorem{theorem}{Theorem}[section]
\newtheorem{corollary}{Corollary}[section]
\newtheorem{remark}{Remark}[section]
\newtheorem{example}{Example}[]
\newtheorem{notation}{Notation Declaration}[]
\newtheorem{question}{Question}[]
\newtheorem{exercise}{Exercise}[section]

%\maketitle




\newpage

\noindent Consider the initial value plroblem (IVP):
\begin{equation}
\left\{
\begin{aligned}
u^{\prime}(t) & =f(t, u(t)), t \in(0, T], \\
u(0) & =u_0.
\end{aligned}
\right .
\label{eq:ivp}
\end{equation}
Discretize the interval $[0,T]$ into $M+1$ equally spaced nodes, forming a grid defined by $t^n =  n h$ for $n=0,1,\cdots,M$, where $h = T/M$.We assume that $f$ is Lipschitz continuous in its second variable uniformly with respect to its first variable and that $u$ is at least three times differentiable.

\setcounter{section}{2}

\setcounter{exercise}{0}


\begin{exercise}
Carry out an error analysis for Backward Euler schema.
\end{exercise}
\begin{proof}[Solution]
Backward Euler schema:
\begin{equation}
\frac{u^{n} - u^{n-1}}{h} = f(t^n, u^n),\quad n = 1,2,\cdots,M.
\label{eq:1-1}
\end{equation}
Its truncation error $R_b^n$:
\begin{equation}
R_b^n = \frac{u(t^n) - u(t^{n-1})}{h} - f(t^n, u(t^n)), \ n=1,\cdots,M.
\label{eq:1-2}
\end{equation}
By the Tylor development:
\[
u(t^{n-1}) = u(t^n) - h u^\prime (t^n) + \frac{h^2}{2} u^{\prime\prime} (\xi^n), \ \text{for some} \ \xi^n\in [t^{n-1}, t^n].
\]
We have $R_b^n = O(h)$ as $h\to 0$.
Let $e^n = u(t^n) - u^n$.
Since \eqref{eq:1-1} and \eqref{eq:1-2}, we have
\[
\frac{e^n - e^{n-1}}{h} - R^n_b = f(t^n, u(t^n))-f(t^n, u^n).
\]
Let $R = \max_{1\leqslant n \leqslant M} |R_b^n|$, and by $|f(t^n, u(t^n))-f(t^n, u^n)|\leqslant L |e^n|$, we have
\[
|e^n| \leqslant |e^{n-1}| + hL |e^n| + h R.
\]
Thus if $h<L^{-1}$,
\[
|e^n| \leqslant \frac{|e^{n-1}|}{1-hL} + \frac{hR}{1-hL} \leqslant \cdots \leqslant 
\frac{|e^{0}|}{(1-hL)^n} + hR \sum_{k=1}^n \frac{1}{(1-hL)^k}
= \frac{|e^{0}|}{(1-hL)^n} + \frac{hR}{hL} [(1-hL)^{-n}-1].
\]
Note that $|e^0|=0$, $(1-hL)^{-n} \leqslant (1-hL)^{-M} = (1-hL)^{-T/h}$, $(1-hL)^{-T/h}$ is non-decreasing over $h\in(0,L^{-1})$, and
\[
\lim_{h\to 0}   (1-hL)^{-T/h} = e^{LT}.
\]
Thus $|e^n| = O(h)$ as $h \to 0$.
\end{proof}



\begin{exercise}
Carry out an error analysis for Crank-Nicolson (Trapezoidal) schema.
\end{exercise}
\begin{proof}[Solution]
Crank-Nicolson schema:
\begin{equation}
\frac{u^{n+1}-u^n}{h}=\frac{f\left(t^{n+1}, u^{n+1}\right)+f\left(t^n, u^n\right)}{2},\ n=0,1,\cdots,M-1.
\label{eq:1-3}
\end{equation}
Its truncation error $R_c^n$:
\begin{equation}
R_c^n = \frac{u(t^{n+1}) - u(t^n)}{h} - \frac{f(t^{n+1},u(t^{n+1})) + f(t^n,u(t^n))}{2}.
\label{eq:1-4}
\end{equation}
By Tylor development:
\[
u(t^{n+1}) = u(t^n) + hu^\prime(t^n) + \frac{h^2}{2} u^{\prime\prime}(t^n) + \frac{h^3}{6} u^{\prime\prime\prime}(\xi^n),\quad \ \text{for some}\ \xi^n \in [t^n, t^{n+1}],
\]
and
\[
u^\prime(t^{n+1}) = u^\prime (t^n) + hu^{\prime\prime}(t^n) + \frac{h^2}{2} u^{\prime\prime\prime}(\eta^n) ,\quad \ \text{for some}\ \eta^n \in [t^n, t^{n+1}],
\]
we have $R_c^n = O(h^2)$ as $h\to 0$.
Let $e^n = u(t^n) - u^n$. Since \eqref{eq:1-3} and \eqref{eq:1-4}, we have
\[
\frac{e^{n+1} - e^n}{h} = \frac{f(t^{n+1}, u(t^{n+1}))-f(t^{n+1}, u^{n+1}) +  f(t^n, u(t^n)) - f(t^n, u^n)}{2} +R^n_c.
\]
Let $R = \max_{0\leqslant n \leqslant M-1} |R_c^n|$, and by $|f(t^n, u(t^n))-f(t^n, u^n)|\leqslant L |e^n|$, we have
\[
|e^{n+1}| \leqslant |e^n| + h R + h \frac{L |e^{n+1}| + L |e^{n}|}{2} .
\]
Hence
\[
|e^{n+1}|  \leqslant \frac{2+hL}{2-hL} |e^n| + h R\leqslant \cdots \leqslant 
\left(\frac{2+hL}{2-hL} \right)^{n+1} |e^0| + h R \frac{1-\left( \frac{2+hL}{2-hL} \right)^{n+1}}{1-\frac{2+hL}{2-hL}}.
\]
Note that $|e^0| = 0$ and
\[
\lim_{h\to 0} \frac{1-\left( \frac{2+hL}{2-hL} \right)^{n+1}}{1-\frac{2+hL}{2-hL}} \leqslant M = T/h.
\]
Thus $|e^n| = O(h^2)$ as $h\to 0$.
\end{proof}


\end{document}

