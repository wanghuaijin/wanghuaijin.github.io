\documentclass[a4paper,11pt]{article}
\usepackage{amsmath, amssymb}
\usepackage{geometry}
\usepackage{graphicx}
\usepackage{ctex}  % 支持中文
%\geometry{a4paper, margin=1in}
\geometry{a4paper, top=0.2in, bottom=0.2in, left=0.8in, right=0.8in}

\usepackage[utf8]{inputenc}
\usepackage[T1]{fontenc}
\usepackage{amsthm}
\usepackage{enumitem}
\usepackage{amssymb}
\usepackage{amsmath}
\usepackage{amsfonts}
\usepackage[version=4]{mhchem}
\usepackage{stmaryrd}
\usepackage{mathrsfs}
\usepackage{bm}
\usepackage{graphicx}
\usepackage[export]{adjustbox}
\graphicspath{ {./images/} }
\usepackage{algorithm}
\usepackage{algorithmic}
\usepackage{makecell}  % 表格换行


\usepackage{hyperref}
\hypersetup{
    colorlinks=true,     % 启用颜色链接
    linkcolor=blue,     % 内部链接的颜色
    citecolor=blue,      % 引用文献的颜色
    urlcolor=blue,       % URL链接的颜色
    linktoc=red,      % 不影响目录链接颜色
}



\title{本科课程:微分方程数值解法}
\author{授课教师: 许传炬教授}
\date{更新时间: \today}

\begin{document}

\maketitle

\vspace{-3em}

\section*{\large 基本信息}
\vspace{-1em}

\begin{itemize}
\item 教材: \href{https://wanghuaijin.github.io/assets/numDEs/LiRonghua.pdf}{李荣华 - 微分方程数值解法} + 幻灯片: \href{https://wanghuaijin.github.io/assets/numDEs/part1.pdf}{part1.pdf}
\vspace{-1em}
\item 上课时间: 周一/周三 3、4 (海韵教学楼301),双周周三 7、8 (海韵实验楼202)
\vspace{-1em}
\item 参考资料: \href{https://wanghuaijin.github.io/assets/numDEs/Alfio1994.pdf}{Numerical Approximation of Partial Different Equations}
\end{itemize}

\vspace{-2em}

\section*{\large 课堂记录}
\vspace{-1em}

(请大家核对, 并及时提交反馈)

\vspace{-1em}
\begin{itemize}
\item 出勤记录: \href{https://wanghuaijin.github.io/assets/numDEs/status/2024-2025_2_0_19020240157508.xls}{出勤记录.xls}
\vspace{-1em}
\item 作业提交记录: \href{https://wanghuaijin.github.io/assets/numDEs/status/homework.xls}{作业记录.xls}
\end{itemize}



\vspace{-2em}

\section*{\large 作业参考答案}
\vspace{-1em}

\begin{itemize}
    \item {第 1 周}:   \href{https://wanghuaijin.github.io/assets/numDEs/exercises/exercise1.pdf}{\texttt{Week1.pdf}}
    \vspace{-1em}
    \item {第 2 周}: \href{https://wanghuaijin.github.io/assets/numDEs/exercises/exercise2.pdf}{\texttt{Week2.pdf}}
    \vspace{-1em}
%    \item {第 3 周}: \texttt{dec24.pdf}, \texttt{dec24.tex}
%    \vspace{-1em}
\end{itemize}



\end{document}

